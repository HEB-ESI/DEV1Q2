\documentclass[11pt,a4paper]{article}
			\usepackage[french]{babel}
					
				\usepackage{pifont}  
				\usepackage[utf8x]{inputenc}
				\usepackage[T1]{fontenc} 
				\usepackage{lmodern}			
				\usepackage{fancyhdr}
				\usepackage{textcomp}
				\usepackage{makeidx}
				\usepackage{tabularx}
				\usepackage{multicol}
				\usepackage{multirow}
				\usepackage{longtable}
				\usepackage{color}
				\usepackage{soul}
				\usepackage{boxedminipage}
				\usepackage{shadow}
				\usepackage{framed}			
				\usepackage{array}
				\usepackage{url}
				\usepackage{ragged2e}
				\usepackage{fancybox}
				\newcommand{\cadretitre}[2]{
				  \vspace*{0.8\baselineskip}
				  \begin{center}%
				  \boxput*(0,1){%
					%\colorbox{white}{\Large\textbf{\ #1\ }}%
				  }%
				  {%
					\setlength{\fboxsep}{10pt}%
				    \Ovalbox{\begin{minipage}{.8\linewidth}\begin{center}\Large\sffamily{#2}\end{center}\end{minipage}}}%
				  \end{center}
				  \vspace*{2\baselineskip}
				  }
			
			\makeatletter
			\def\@seccntformat#1{\protect\makebox[0pt][r]{\csname the#1\endcsname\quad}}
			\makeatother

				% Permet d'afficher qqchose à une positin absolue
				\usepackage[absolute]{textpos}
				\setlength{\TPHorizModule}{1cm}
				\setlength{\TPVertModule}{\TPHorizModule}
	
				\usepackage[titles]{tocloft}
				\setlength{\cftbeforesecskip}{0.5ex}
				\setlength{\cftbeforesubsecskip}{0.2ex}
				\addto\captionsfrench{\renewcommand\contentsname{}}
				
				\usepackage[font=scriptsize]{caption}
				
				\usepackage{listings}
\lstdefinestyle{lstverb}
  {
    basicstyle=\footnotesize,
    frameround=tttt, frame=trbl, framerule=0pt, rulecolor=\color{gray},
    lineskip=-1pt,   % pour rapprocher les lignes
    flexiblecolumns, escapechar=\\,
    tabsize=4, extendedchars=true
  }
\lstnewenvironment{Java}[1][]{\lstset{style=lstverb,language=java,#1}}{}
				\ifx\pdfoutput\undefined
					\usepackage{graphicx}
				\else
					\usepackage[pdftex]{graphicx}
				\fi
				\usepackage[a4paper, hyperfigures=true, colorlinks, linkcolor=black, citecolor=blue,urlcolor=blue, pagebackref=true, bookmarks=true, bookmarksopen=true,bookmarksnumbered=true,
                pdfauthor={}, pdftitle={TD Séquentiel - Rappels de base}, pdfkeywords={TD S\'equentiel - Rappels de base, },pdfpagemode=UseOutlines,pdfpagetransition=Dissolve,nesting=true,
				backref, pdffitwindow=true, bookmarksnumbered=true]{hyperref}
				\usepackage{supertabular}
				\usepackage[table]{xcolor}
				\usepackage{url}
				\usepackage{caption} 
				\setlength{\parskip}{1.3ex plus 0.2ex minus 0.2ex}
				\setlength{\parindent}{0pt}
				
				\makeatletter
				\def\url@leostyle{ \@ifundefined{selectfont}{\def\UrlFont{\sf}}{\def\UrlFont{\footnotesize\ttfamily}}}
				\makeatother
				\urlstyle{leo}
				
				\definecolor{examplecolor}{rgb}{0.156,0.333,0.443}
				\definecolor{definitioncolor}{rgb}{0.709,0.784,0.454}
				\definecolor{exercisecolor}{rgb}{0.49,0.639,0}
				\definecolor{hintcolor}{rgb}{0.941,0.674,0.196}
				\definecolor{tableHeadercolor}{rgb}{0.709,0.784,0.454}
				\definecolor{tablerowAltcolor}{rgb}{.866,.905,.737}
				\definecolor{tablerowAlt2color}{rgb}{.968,.976,.933}
				\definecolor{verylightgray}{rgb}{0.98,0.98,0.98}
				
				\newenvironment{fshaded}{
				\def\FrameCommand{\fcolorbox{framecolor}{shadecolor}}
				\MakeFramed {\FrameRestore}}
				{\endMakeFramed}
				
				\newenvironment{fexample}[1][]{\definecolor{shadecolor}{rgb}{.913,.913,.913}
				\definecolor{framecolor}{rgb}{.156,.333,.443}
				\begin{fshaded}}{\end{fshaded}} 
				
				\newenvironment{fdefinition}{\definecolor{shadecolor}{rgb}{.913,.913,.913}
				\definecolor{framecolor}{rgb}{.709,.784,.454}
				\begin{fshaded}}{\end{fshaded}}
				
				\newenvironment{fexercise}{\definecolor{shadecolor}{rgb}{.913,.913,.913}
				\definecolor{framecolor}{rgb}{.49,.639,0}
				\begin{fshaded}}{\end{fshaded}}
				
				\newenvironment{fhint}{\definecolor{shadecolor}{rgb}{.913,.913,.913}
				\definecolor{framecolor}{rgb}{.941,.674,.196}
				\begin{fshaded}}{\end{fshaded}}	
				
				\newcommand{\PreserveBackslash}[1]{
				\let\temp=\\#1\let\\=\temp
				}
				\let\PBS=\PreserveBackslash
				\newcolumntype{A}{>{\PBS\raggedright\small\hspace{0pt}}X}
				\newcolumntype{L}[1]{>{\PBS\raggedright\small\hspace{0pt}}p{#1}}
				\newcolumntype{R}[1]{>{\PBS\raggedleft\small\hspace{0pt}}p{#1}}
				\newcolumntype{C}[1]{>{\PBS\centering\small\hspace{0pt}}p{#1}}
				
				\makeindex
				
				\title{TD S\'equentiel - Rappels de base}	
			\date{}
			\author{\scriptsize{}}
			\definecolor{light-gray}{gray}{0.8}
			\renewcommand{\headrulewidth}{0pt}
			\fancyhead[L]{
				\footnotesize\textsc{Haute \'Ecole de Bruxelles}\\
	    			\footnotesize\textsc{\'Ecole Sup\'erieure d'Informatique}
			}
			\fancyhead[R]{
				\footnotesize{Bachelor en Informatique}\\
				\footnotesize{Laboratoires Java} - 
			\footnotesize{1\`ere ann\'ee}}
				\fancyfoot[L]{ }
				\fancyfoot[C]{}
				\fancyfoot[R]{\scriptsize{\textcolor{gray}{version 2014-2015 (\today)}}}
				\pagestyle{plain}
				\reversemarginpar
				\usepackage{rotating}						
				\begin{document}
					\begin{textblock}{9}(2,3.2)
						\includegraphics[width=2cm]{../../../_templates/java/icons/logo-esi}
					\end{textblock}
				
				
				
				
				%\maketitle
				\cadretitre{TD1}{TD S\'equentiel - Rappels de base}
				\thispagestyle{fancy}
        \marginpar{\begin{sideways}
            \begin{minipage}[t]{1cm}
            \begin{tiny}
            \includegraphics[width=1\linewidth,height=1\textheight,keepaspectratio=true]{../../../_templates/java/icons/cc-gris.jpg}
			\end{tiny}
			\end{minipage}
            \begin{minipage}[b]{19cm}
            \begin{tiny}
            \textcolor{gray}{Distribué sous licence Creative Commons Paternité - Partage à l'Identique 2.0 Belgique 
            (\texttt{http://creativecommons.org/licenses/by-sa/2.0/be/})
			\vspace{-1em}
			\\Les autorisations au-delà du champ de cette licence peuvent être obtenues à 
			\texttt{http://www.heb.be/esi}
			- \texttt{mcodutti@heb.be}
			}\end{tiny}
			\end{minipage}
        \end{sideways}}
            \begin{abstract}
			Ce TD a pour but de 
			
            \par
        
            \par
        \end{abstract}
				\vspace{-2em}\tableofcontents
				\pagestyle{plain}
            \clearpage
            \fancyhead[L,C,R]{}
            \fancyfoot[L,C]{}
            \fancyfoot[R]{ \scriptsize{\textcolor{gray}{
				TDSeq - page \thepage}}}
				\thispagestyle{fancy}
				\pagestyle{fancy}
	   
            \section{Algorithmes s\'equentiels}Revoyons ici les bases du pseudo-code et leur traduction en Java.\subsection{Le pseudo-code}
		    Le pseudo-code ou Langage de Description des Algorithmes (LDA en abr\'eg\'e) est un langage
        formel et symbolique utilisant :
        
					\begin{itemize}
				
			\item 
            des noms symboliques destin\'es \`a repr\'esenter les objets sur lesquels s'effectuent des
            actions ;
          
			\item 
            des op\'erateurs symboliques ou des mots-cl\'es traduisant les op\'erations primitives
            ex\'ecutables par un ex\'ecutant donn\'e ;
          
			\item 
            des structures de contr\^ole types.
          
					\end{itemize}
				
            \par
        \section{Variables et types}
				Nous savons que les op\'erations que l'ordinateur devra ex\'ecuter portent sur des \'el\'ements qui
        sont les \textbf{donn\'ees} du probl\`eme.
      
            \par
        
        Lorsqu'on attribue un \textbf{nom} 
        et un \textbf{type} \`a ces donn\'ees, on
        parle alors de \textbf{variables}. 
      
            \par
        
        Dans un algorithme, une variable conserve toujours son nom et
        son type, mais peut changer de \textbf{valeur}.
         
					\begin{itemize}
				
			\item 
            Le \textbf{nom} d'une variable permet de la caract\'eriser et de la reconnaitre ;
          
			\item 
            le \textbf{type} d'une variable d\'ecrit la nature de son contenu.
          
					\end{itemize}
				
            \par
        \subsection{Les types autoris\'es en algo}
          Dans un premier temps, les seuls types utilis\'es sont :
          
					\begin{itemize}
				
			\item \verb@entier@ pour les nombres entiers ;
			\item \verb@réel@ pour les nombres r\'eels ;
			\item \verb@caractère@ pour les diff\'erentes lettres et caract\`eres
              (par exemple ceux qui apparaissent sur un clavier : 'a', '1', '\#', etc.)
			\item \verb@chaine@ pour les variables contenant un ou plusieurs caract\`ere(s) ou aucun (la chaine vide)
              (par exemple : "Bonjour", "Bonjour le monde", "a", "", etc.)
			\item \verb@booléen@ les variables de ce type 
            ne peuvent valoir que \verb@vrai@ ou \verb@faux@
					\end{itemize}
				
            \par
        
			
		\subparagraph{Le type des donn\'ees} 
		
                \textcolor{white}{.} \par
             
								Quel(s) type(s) de donn\'ees utiliseriez-vous pour repr\'esenter :
							
					\begin{itemize}
				
			\item une date du calendrier ?  \textcolor{gray}{\underline{\hspace*{10em}}} 
			\item un moment dans la journ\'ee ?  \textcolor{gray}{\underline{\hspace*{10em}}} 
			\item le prix d'un produit en grande surface ? \textcolor{gray}{\underline{\hspace*{5em}}} 
			\item votre nom ?  \textcolor{gray}{\underline{\hspace*{10em}}} 
			\item vos initiales ? \textcolor{gray}{\underline{\hspace*{10em}}} 
			\item votre adresse ?  \textcolor{gray}{\underline{\hspace*{10em}}} 
					\end{itemize}
				\subsection{Les types \'equivalents en java}
          Les \'equivalents Java des types donn\'es en algo sont
          
					\begin{itemize}
				
			\item \verb@int@ pour le type \verb@entier@ pour les nombres entiers ;
			\item \verb@double@pour le type \verb@réel@ pour les nombres r\'eels ;
			\item \verb@char@ pour le type \verb@caractère@ pour les diff\'erentes lettres et caract\`eres
			\item \verb@String@ pour le type \verb@chaine@ 
              pour les variables contenant un ou plusieurs caract\`ere(s) ou aucun (la chaine vide)
            
			\item \verb@boolean@ pour le type \verb@booléen@ les variables de ce type 
            ne peuvent valoir, en Java, que \verb@true@ ou \verb@false@
					\end{itemize}
				
            \par
        
			
		\subparagraph{Les commandes de base} 
		
                \textcolor{white}{.} \par
             
								Quel(s) type(s) de donn\'ees utiliseriez-vous pour repr\'esenter :
							
					\begin{itemize}
				
			\item une date du calendrier ?  \textcolor{gray}{\underline{\hspace*{10em}}} 
			\item un moment dans la journ\'ee ?  \textcolor{gray}{\underline{\hspace*{10em}}} 
			\item le prix d'un produit en grande surface ? \textcolor{gray}{\underline{\hspace*{5em}}} 
			\item votre nom ?  \textcolor{gray}{\underline{\hspace*{10em}}} 
			\item vos initiales ? \textcolor{gray}{\underline{\hspace*{10em}}} 
			\item votre adresse ?  \textcolor{gray}{\underline{\hspace*{10em}}} 
					\end{itemize}
				\subsection{D\'eclaration de variables en algo}
          La \textbf{d\'eclaration} d'une variable est l'instruction qui d\'efinit son \textbf{nom} 
          et son \textbf{type}.
        
            \par
        \,\verb|num1, num2 : entiers|\,
            \par
        
          L'ensemble des instructions de la forme \,\verb|variable1, variable2,. . . : type|\,
          forme la partie d'un algorithme nomm\'ee \textbf{d\'eclaration des variables}. 
        
            \par
        
          La d\'eclaration des informations apparaitra toujours en \textbf{d\'ebut} d'algorithme, ou dans un bloc annexe appel\'e
          dictionnaire des variables ou encore dictionnaire des donn\'ees.
        
            \par
        
			
		\subparagraph{Attention} 
		
					\textcolor{white}{.} \par
				
            \par
        
          Attention, lors de la d\'eclaration d'une variable, \textbf{celle-ci n'a pas de valeur} ! 
          Nous verrons que c'est l'instruction d'affectation qui va servir \`a donner un contenu aux variables d\'eclar\'ees.
        
            \par
        
			
		\subparagraph{Comment nommer correctement une variable ?} 
		
					\textcolor{white}{.} \par
				
            \par
        
          Le but est de trouver un nom qui soit suffisamment \textbf{court}, 
          tout en restant \textbf{explicite} et ne pr\^etant \textbf{pas \`a confusion}.
        
            \par
        
          Ainsi \textit{num1} est plus appropri\'e pour d\'esigner le premier num\'erateur 
          que \textit{zozo1}, \textit{tintin}, \textit{bidule}
          ou \textit{premierNum\'erateur}. 
          De m\^eme, ne pas appeler \textit{den} la variable repr\'esentant le num\'erateur.
        
            \par
        
          Il faut aussi tenir compte que les langages de programmation imposent certaines limitations
          (parfois diff\'erentes d'un langage \`a l'autre) ce qui peut n\'ecessiter une modification du nom
          lors de la traduction.
        
            \par
        
          Voici quelques r\`egles et limitations traditionnelles dans les langages de programmation :
          
					\begin{itemize}
				
			\item 
            Un nom de variable est g\'en\'eralement une suite de caract\`eres alphanum\'eriques d'un
            seul tenant (pas de caract\`eres blancs) et ne commen\c cant jamais par un chiffre. Ainsi
            x1 est correct mais pas 1x.
            
			\item 
            Pour donner un nom compos\'e \`a une variable, on peut utiliser le \guillemotleft  tiret bas \guillemotright  ou underscore 
            (par ex. premier\_num\'erateur) mais on d\'econseille d'utiliser le signe \guillemotleft  - \guillemotright  qui est
            plut\^ot r\'eserv\'e \`a la soustraction. Ainsi, dans la plupart des langages, premier-num\'erateur
            serait interpr\'et\'e comme la soustraction des variables premier et num\'erateur.
            
			\item 
            Une alternative \`a l'utilisation du tiret bas pour l'\'ecriture de noms de variables compos\'es
            est la notation \guillemotleft  chameau \guillemotright  (camelCase en anglais), qui consiste \`a mettre une majuscule
            au d\'ebut des mots (g\'en\'eralement \`a partir du deuxi\`eme), par exemple premierNombre
            ou dateNaissance.
            
			\item 
            Les indices et exposants sont proscrits.
            
			\item 
            Les mots-cl\'es du langage sont interdits (par exemple for, if, while pour Java et Cobol)
            et on d\'econseille d'utiliser les mots-cl\'es du pseudo-code (tels que Lire, Afficher, pour. . .)
            
			\item 
            Certains langages n'autorisent pas les caract\`eres accentu\'es (tels que \`a, \c c, \^e, \o, etc.)
            ou les lettres des alphabets non latins mais d'autres oui ; certains font la
            distinction entre les minuscules et majuscules, d'autres non. En algorithmique, nous
            admettons, dans les noms de variables les caract\`eres accentu\'es du fran\c cais, par ex. :
            dur\'ee, int\'er\^ets, etc.
            
					\end{itemize}
				
            \par
        
			
		\subparagraph{Comment d\'eclarer} 
		
                \textcolor{white}{.} \par
            
							  Quelle instruction permet de d\'eclarer :
							
					\begin{itemize}
				
			\item le jour d'une date ? \textcolor{gray}{\underline{\hspace*{10em}}} 
			\item l'heure d'un moment ?  \textcolor{gray}{\underline{\hspace*{10em}}} 
			\item le prix d'un produit en grande surface ? \textcolor{gray}{\underline{\hspace*{10em}}} 
			\item votre nom ?  \textcolor{gray}{\underline{\hspace*{10em}}} 
					\end{itemize}
				\subsection{D\'eclaration de variables en Java}
          La \textbf{d\'eclaration} d'une variable est l'instruction qui d\'efinit son \textbf{nom} 
          et son \textbf{type}.
        
            \par
        \,\verb|int num1;|\,
            \par
        
          L'ensemble des instructions de la forme \,\verb|type nom;|\,
          forme la \textbf{d\'eclaration des variables}. 
        
            \par
        
          La d\'eclaration des informations apparaitra en \textbf{d\'ebut} de bloc.
        
            \par
        
			
		\subparagraph{Attention} 
		
					\textcolor{white}{.} \par
				
            \par
        
          Attention, comme en algo, lors de la d\'eclaration d'une variable, \textbf{celle-ci n'a pas de valeur} ! 
          Nous verrons que c'est l'instruction d'affectation qui va servir \`a donner un contenu aux variables d\'eclar\'ees.
        
            \par
        
			
		\subparagraph{Comment nommer correctement une variable ?} 
		
					\textcolor{white}{.} \par
				
            \par
        
          Le but est de trouver un nom qui soit suffisamment \textbf{court}, 
          tout en restant \textbf{explicite} et ne pr\^etant \textbf{pas \`a confusion}.
        
            \par
        
          Il faut aussi tenir compte des limitations du langage.
          
					\begin{itemize}
				
			\item 
            Un nom de variable est g\'en\'eralement une suite de caract\`eres alphanum\'eriques d'un
            seul tenant (pas de caract\`eres blancs) et ne commen\c cant jamais par un chiffre. Ainsi
            x1 est correct mais pas 1x.
            
			\item 
            un nom de variable ne peut comporter que des lettres, des chiffres, les caract\`eres \textit{\_} et \textit{\$}.
            
			\item 
            L'\'ecriture de noms de variables compos\'es
            est la notation \guillemotleft  chameau \guillemotright  (camelCase en anglais), qui consiste \`a mettre une majuscule
            au d\'ebut des mots (g\'en\'eralement \`a partir du deuxi\`eme), par exemple premierNombre
            ou dateNaissance.
            
			\item 
            Pour donner un nom compos\'e \`a une variable enti\`erement en majuscules, on peut utiliser le \guillemotleft  tiret bas \guillemotright  ou underscore 
            (par ex. PREMIER\_NUMERATEUR).
            
			\item 
            Les mots-cl\'es du langage sont interdits (par exemple for, if, while)
            
					\end{itemize}
				
            \par
        
			
		\subparagraph{Comment d\'eclarer} 
		
                \textcolor{white}{.} \par
            
							  Quelle instruction permet de d\'eclarer :
							
					\begin{itemize}
				
			\item le jour d'une date ? \textcolor{gray}{\underline{\hspace*{10em}}} 
			\item l'heure d'un moment ?  \textcolor{gray}{\underline{\hspace*{10em}}} 
			\item le prix d'un produit en grande surface ? \textcolor{gray}{\underline{\hspace*{10em}}} 
			\item votre nom ?  \textcolor{gray}{\underline{\hspace*{10em}}} 
					\end{itemize}
				\section{Op\'erateurs et expressions}
				Les \textbf{op\'erateurs} agissent sur les \textbf{variables} 
				et les \textbf{constantes} pour former des \textbf{expressions}. 
      
            \par
        
				Une expression est donc une combinaison coh\'erente de variables, de constantes et d'op\'erateurs,
				\'eventuellement accompagn\'es de parenth\`eses.
			
            \par
        \subsection{Les op\'erateurs arithm\'etiques \'el\'ementaires en algo}
          Ce sont les op\'erateurs binaires bien connus :
          
					\begin{itemize}
				
			\item \verb@+@ addition
			\item \verb@-@ soustraction
			\item \verb@*@ multiplication
			\item \verb@/@ division r\'eelle
					\end{itemize}
				
          Ils agissent sur des variables ou expressions \`a valeurs enti\`eres ou r\'eelles. 
        
            \par
        
          Plusieurs op\'erateurs peuvent \^etre utilis\'es pour former des expressions plus ou moins complexes, 
          en tenant compte des r\`egles de calcul habituelles, notamment la priorit\'e de la multiplication et de la division
          sur l'addition et la soustraction. 
        
            \par
        
          Il est aussi permis d'utiliser des parenth\`eses, par exemple \,\verb|a – (b + c * d)/x|\,. 
        
            \par
        
          Tout emploi de la division devra \^etre accompagn\'e d'une r\'eflexion sur la
          valeur du d\'enominateur, une division par 0 entrainant toujours l'arr\^et d'un algorithme.
        
            \par
        \subsection{Les op\'erateurs arithm\'etiques \'el\'ementaires en Java}
          Ce sont les op\'erateurs binaires bien connus :
          
					\begin{itemize}
				
			\item \verb@+@ addition
			\item \verb@-@ soustraction
			\item \verb@*@ multiplication
			\item \verb@/@ division r\'eelle si au moins un des deux op\'erandes est r\'eel
			\item \verb@/@ division enti\`ere si les deux op\'erandes sont entiers
					\end{itemize}
				
          Ils agissent sur des variables ou expressions \`a valeurs enti\`eres ou r\'eelles. 
        
            \par
        
          Plusieurs op\'erateurs peuvent \^etre utilis\'es pour former des expressions plus ou moins complexes, 
          en tenant compte des r\`egles de calcul habituelles, notamment la priorit\'e de la multiplication et de la division
          sur l'addition et la soustraction. 
        
            \par
        
          Il est aussi permis d'utiliser des parenth\`eses, par exemple \,\verb|a – (b + c * d)/x|\,. 
        
            \par
        
          Tout emploi de la division devra \^etre accompagn\'e d'une r\'eflexion sur la
          valeur du d\'enominateur, une division par 0 entrainant un arr\^et du programme.
        
            \par
        \subsection{Les op\'erateurs DIV et MOD en algo}
          Ce sont deux op\'erateurs tr\`es importants qui ne peuvent s'utiliser qu'avec des variables enti\`eres :
          
					\begin{itemize}
				
			\item \verb@DIV@ division enti\`ere
			\item \verb@MOD@ reste de la division enti\`ere
					\end{itemize}
				
            \par
        \subsection{Les op\'erateurs DIV et MOD en Java}
          Ce sont deux op\'erateurs tr\`es importants qui ne peuvent s'utiliser qu'avec des variables enti\`eres :
          
					\begin{itemize}
				
			\item \verb@/@ division enti\`ere si les deux op\'erandes sont entiers
			\item \verb@%@ reste de la division enti\`ere
					\end{itemize}
				
            \par
        \subsection{Les fonctions math\'ematiques complexes en algo}
					\begin{itemize}
				
			\item L'\'el\'evation \`a la puissance sera not\'ee ** ou ˆ . 
			\item Pour la racine carr\'ee d'une variable x nous \'ecrirons √x . Attention, pour ce dernier, de veiller \`a ne l'utiliser qu'avec un radicant positif !
            Exemple : \verb@(−b +√(b ∗ ∗2 − 4 ∗ a ∗ c))/(2 ∗ a)@
			\item  \`A votre avis, pourquoi ne pas avoir \'ecrit \guillemotleft  4ac \guillemotright  et \guillemotleft  2a \guillemotright  ?
			\item Si n\'ecessaire, on se permettra d'utiliser les autres fonctions math\'ematiques sous leur forme
            la plus courante dans la plupart des langages de programmation (exemples : sin(x), tan(x), log(x), exp(x). . .)
					\end{itemize}
				
            \par
        \subsection{Les fonctions math\'ematiques complexes en Java}
        L'essentiel des fonctions math\'ematiques se trouvent dans la classe Math.
          
					\begin{itemize}
				
			\item Math.pow(a,b) est l'\'el\'evation \`a la puissance aˆb
			\item Math.sqrt(x) est la racine carr\'ee d'une variable x, √x . Attention, pour ce dernier, de veiller \`a ne l'utiliser qu'avec un radicant positif !
			\item Si n\'ecessaire,les autres fonctions math\'ematiques Math.sin(x), Math.tan(x), Math.log(x), Math.exp(x). . .)
					\end{itemize}
				
            \par
        \section{L'affectation d'une valeur \`a une variable}
				Cette op\'eration est probablement l'op\'eration la plus importante. En effet, une variable ne
        prend son sens r\'eel que si elle re\c coit \`a un moment donn\'e une valeur. Il y a deux moyens de
        donner une valeur \`a une variable.
      
            \par
        \subsection{Affectation interne en algo}
          On parle d'\textbf{affectation interne} lorsque la valeur d'une variable est \guillemotleft  calcul\'ee \guillemotright  par l'ex\'ecutant
          de l'algorithme lui-m\^eme \`a partir de donn\'ees qu'il connait d\'ej\`a :
          \,\verb|nomVariable ← expression|\,
          Rappelons qu'une expression est une combinaison de variables et d'op\'erateurs. \textbf{L'expression a une valeur}.
        
            \par
        
			
		\subparagraph{} 
		
                \textcolor{white}{.} \par
            Les exemples d'affectation sont-ils corrects ?
						
            \begin{itemize} 
        
            \item[ \ding{"6F} ] somme ← nombre1 + nombre2
        
            \item[ \ding{"6F} ] denRes ← den1 * den2
        
            \item[ \ding{"6F} ] cpt ← cpt + 1
        
            \item[ \ding{"6F} ] delta ← b**2 - 4*a*c
        
            \item[ \ding{"6F} ] maChaine ← "Bonjour"
        
            \item[ \ding{"6F} ] test ← a = b
        
            \item[ \ding{"6F} ] somme + 1 ← 3
        
            \item[ \ding{"6F} ] somme ← 3n
        
            \end{itemize} 
        
				Remarques
				
					\begin{itemize}
				
			\item 
              Il est de r\`egle que le r\'esultat de l'expression \`a droite du signe d'affectation (←) soit de
              m\^eme type que la variable \`a sa gauche. On tol\`ere certaines exceptions :
              
					\begin{itemize}
				
			\item 
                  varEnti\`ere ← varR\'eelle : dans ce cas le contenu de la variable sera la valeur tronqu\'ee de l'expression r\'eelle. 
                  Par exemple si \guillemotleft  nb \guillemotright  est une variable de type entier,
                  son contenu apr\`es l'instruction \,\verb|nb ← 15/4 »|\, sera 3
                
			\item varR\'eelle ← varEnti\`ere : ici, il n'y a pas de perte de valeur.
			\item varChaine ← varCaract\`ere : \'equivalent \`a varChaine ← chaine(varCaract\`ere).
			\item Le contraire n'est \'evidemment pas accept\'e.
					\end{itemize}
				
			\item 
            Seules les variables d\'eclar\'ees peuvent \^etre affect\'ees, que ce soit par l'affectation externe
            ou interne !
            
			\item 
            Nous ne m\'elangerons pas la d\'eclaration d'une variable et son affectation interne dans
            une m\^eme ligne de code, donc pas d'instructions hybrides du genre x ← 2 : entier ou
            encore x : entier(0).
            
			\item 
            Pour l'affectation interne, toutes les variables apparaissant dans l'expression doivent
            avoir \'et\'e affect\'ees pr\'ealablement. Le contraire provoquerait un arr\^et de l'algorithme.
            
					\end{itemize}
				
            \par
        \subsection{Affectation interne en Java}
          On parle d'\textbf{affectation interne} lorsque la valeur d'une variable est \guillemotleft  calcul\'ee \guillemotright  par l'ex\'ecutant
          de l'algorithme lui-m\^eme \`a partir de donn\'ees qu'il connait d\'ej\`a :
          \,\verb|nomVariable = expression;|\,
          Rappelons qu'une expression est une combinaison de variables et d'op\'erateurs. \textbf{L'expression a une valeur}.
        
            \par
        
			
		\subparagraph{} 
		
                \textcolor{white}{.} \par
            Les exemples d'affectation sont-ils corrects ?
						
            \begin{itemize} 
        
            \item[ \ding{"6F} ] somme = nombre1 + nombre2;
        
            \item[ \ding{"6F} ] denRes = den1 * den2;
        
            \item[ \ding{"6F} ] cpt = cpt + 1;
        
            \item[ \ding{"6F} ] delta = b**2 - 4*a*c;
        
            \item[ \ding{"6F} ] maChaine = "Bonjour";
        
            \item[ \ding{"6F} ] somme + 1 = 3;
        
            \item[ \ding{"6F} ] somme = 3n;
        
            \end{itemize} 
        \subsection{Affectation externe en algo}
          On parle d'\textbf{affectation externe} lorsque la valeur \`a affecter \`a une variable
           est donn\'ee par l'utilisateur qui la communique \`a l'ex\'ecutant quand celui-ci le lui demande : cette valeur est
          donc externe \`a la proc\'edure (l'ordinateur ne peut la deviner lui-m\^eme !)
        
            \par
        
          L'affectation externe est donc la primitive qui permet de recevoir de l'utilisateur, au moment
          o\`u l'algorithme se d\'eroule, une ou plusieurs valeur(s) et de les affecter \`a des variables en
          m\'emoire. 
        
            \par
        
          Nous noterons :
          \,\verb|lire liste_de_variables_à_lire|\,
            \par
        
          Exemples
          \,\verb|lire nombre1, nombre2|\,\,\verb|lire num1, den1, num2, den2|\,
            \par
        
          L'ex\'ecution de cette instruction provoque une pause dans le d\'eroulement de l'algorithme ;
          l'ex\'ecutant demande alors \`a l'utilisateur les valeurs des variables \`a lire. Ces valeurs viennent
          donc de l'ext\'erieur ; une fois introduites dans le syst\`eme, elles sont affect\'ees aux variables
          concern\'ees et l'algorithme peut reprendre son cours. Les possibilit\'es d'introduction de don-
          n\'ees sont nombreuses : citons par exemple l'encodage de donn\'ees au clavier, un clic de souris,
          le toucher d'un \'ecran tactile, des donn\'ees provenant d'un fichier, etc.
        
            \par
        \subsection{Affectation externe en Java}
          On parle d'\textbf{affectation externe} lorsque la valeur \`a affecter \`a une variable
           est donn\'ee par l'utilisateur qui la communique \`a l'ex\'ecutant quand celui-ci le lui demande : cette valeur est
          donc externe \`a la proc\'edure (l'ordinateur ne peut la deviner lui-m\^eme !)
        
            \par
        
          L'affectation externe est donc la primitive qui permet de recevoir de l'utilisateur, au moment
          o\`u l'algorithme se d\'eroule, une ou plusieurs valeur(s) et de les affecter \`a des variables en
          m\'emoire. 
        
            \par
        
          Nous noterons :
          
            \par
        \begin{Java}
 import java. util .Scanner;
// ...
Scanner clavier = new Scanner(System.in);
// ...
int nombre1 = clavier. nextInt ();
				\end{Java}
          Les diff\'erentes lectures possibles sont :
          
					\begin{itemize}
				
			\item pour un entier : clavier.nextInt();
			\item pour un r\'eel : clavier.nextDouble();
			\item pour un bool\'een : clavier.nextBoolean();
			\item pour un mot : clavier.next()
			\item pour une ligne : clavier.nextLine();
			\item pour un caract\`ere : clavier.next.charAt(0);
					\end{itemize}
				
            \par
        
				\end{document}
			