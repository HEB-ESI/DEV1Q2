\documentclass[11pt,a4paper]{article}
			\usepackage[french]{babel}
					
				\usepackage{pifont}  
				\usepackage[utf8x]{inputenc}
				\usepackage[T1]{fontenc} 
				\usepackage{lmodern}			
				\usepackage{fancyhdr}
				\usepackage{textcomp}
				\usepackage{makeidx}
				\usepackage{tabularx}
				\usepackage{multicol}
				\usepackage{multirow}
				\usepackage{longtable}
				\usepackage{color}
				\usepackage{soul}
				\usepackage{boxedminipage}
				\usepackage{shadow}
				\usepackage{framed}			
				\usepackage{array}
				\usepackage{url}
				\usepackage{ragged2e}
				\usepackage{fancybox}
				\newcommand{\cadretitre}[2]{
				  \vspace*{0.8\baselineskip}
				  \begin{center}%
				  \boxput*(0,1){%
					%\colorbox{white}{\Large\textbf{\ #1\ }}%
				  }%
				  {%
					\setlength{\fboxsep}{10pt}%
				    \Ovalbox{\begin{minipage}{.8\linewidth}\begin{center}\Large\sffamily{#2}\end{center}\end{minipage}}}%
				  \end{center}
				  \vspace*{2\baselineskip}
				  }
			
			\makeatletter
			\def\@seccntformat#1{\protect\makebox[0pt][r]{\csname the#1\endcsname\quad}}
			\makeatother

				% Permet d'afficher qqchose à une positin absolue
				\usepackage[absolute]{textpos}
				\setlength{\TPHorizModule}{1cm}
				\setlength{\TPVertModule}{\TPHorizModule}
	
				\usepackage[titles]{tocloft}
				\setlength{\cftbeforesecskip}{0.5ex}
				\setlength{\cftbeforesubsecskip}{0.2ex}
				\addto\captionsfrench{\renewcommand\contentsname{}}
				
				\usepackage[font=scriptsize]{caption}
				
				\usepackage{listings}
\lstdefinestyle{lstverb}
  {
    basicstyle=\footnotesize,
    frameround=tttt, frame=trbl, framerule=0pt, rulecolor=\color{gray},
    lineskip=-1pt,   % pour rapprocher les lignes
    flexiblecolumns, escapechar=\\,
    tabsize=4, extendedchars=true
  }
\lstnewenvironment{Java}[1][]{\lstset{style=lstverb,language=java,#1}}{}
				\ifx\pdfoutput\undefined
					\usepackage{graphicx}
				\else
					\usepackage[pdftex]{graphicx}
				\fi
				\usepackage[a4paper, hyperfigures=true, colorlinks, linkcolor=black, citecolor=blue,urlcolor=blue, pagebackref=true, bookmarks=true, bookmarksopen=true,bookmarksnumbered=true,
                pdfauthor={}, pdftitle={TD Séquentiel - Rappels de base}, pdfkeywords={TD S\'equentiel - Rappels de base, },pdfpagemode=UseOutlines,pdfpagetransition=Dissolve,nesting=true,
				backref, pdffitwindow=true, bookmarksnumbered=true]{hyperref}
				\usepackage{supertabular}
				\usepackage[table]{xcolor}
				\usepackage{url}
				\usepackage{caption} 
				\setlength{\parskip}{1.3ex plus 0.2ex minus 0.2ex}
				\setlength{\parindent}{0pt}
				
				\makeatletter
				\def\url@leostyle{ \@ifundefined{selectfont}{\def\UrlFont{\sf}}{\def\UrlFont{\footnotesize\ttfamily}}}
				\makeatother
				\urlstyle{leo}
				
				\definecolor{examplecolor}{rgb}{0.156,0.333,0.443}
				\definecolor{definitioncolor}{rgb}{0.709,0.784,0.454}
				\definecolor{exercisecolor}{rgb}{0.49,0.639,0}
				\definecolor{hintcolor}{rgb}{0.941,0.674,0.196}
				\definecolor{tableHeadercolor}{rgb}{0.709,0.784,0.454}
				\definecolor{tablerowAltcolor}{rgb}{.866,.905,.737}
				\definecolor{tablerowAlt2color}{rgb}{.968,.976,.933}
				\definecolor{verylightgray}{rgb}{0.98,0.98,0.98}
				
				\newenvironment{fshaded}{
				\def\FrameCommand{\fcolorbox{framecolor}{shadecolor}}
				\MakeFramed {\FrameRestore}}
				{\endMakeFramed}
				
				\newenvironment{fexample}[1][]{\definecolor{shadecolor}{rgb}{.913,.913,.913}
				\definecolor{framecolor}{rgb}{.156,.333,.443}
				\begin{fshaded}}{\end{fshaded}} 
				
				\newenvironment{fdefinition}{\definecolor{shadecolor}{rgb}{.913,.913,.913}
				\definecolor{framecolor}{rgb}{.709,.784,.454}
				\begin{fshaded}}{\end{fshaded}}
				
				\newenvironment{fexercise}{\definecolor{shadecolor}{rgb}{.913,.913,.913}
				\definecolor{framecolor}{rgb}{.49,.639,0}
				\begin{fshaded}}{\end{fshaded}}
				
				\newenvironment{fhint}{\definecolor{shadecolor}{rgb}{.913,.913,.913}
				\definecolor{framecolor}{rgb}{.941,.674,.196}
				\begin{fshaded}}{\end{fshaded}}	
				
				\newcommand{\PreserveBackslash}[1]{
				\let\temp=\\#1\let\\=\temp
				}
				\let\PBS=\PreserveBackslash
				\newcolumntype{A}{>{\PBS\raggedright\small\hspace{0pt}}X}
				\newcolumntype{L}[1]{>{\PBS\raggedright\small\hspace{0pt}}p{#1}}
				\newcolumntype{R}[1]{>{\PBS\raggedleft\small\hspace{0pt}}p{#1}}
				\newcolumntype{C}[1]{>{\PBS\centering\small\hspace{0pt}}p{#1}}
				
				\makeindex
				
				\title{TD S\'equentiel - Rappels de base}	
			\date{}
			\author{\scriptsize{}}
			\definecolor{light-gray}{gray}{0.8}
			\renewcommand{\headrulewidth}{0pt}
			\fancyhead[L]{
				\footnotesize\textsc{Haute \'Ecole de Bruxelles}\\
	    			\footnotesize\textsc{\'Ecole Sup\'erieure d'Informatique}
			}
			\fancyhead[R]{
				\footnotesize{Bachelor en Informatique}\\
				\footnotesize{Laboratoires Java} - 
			\footnotesize{1\`ere ann\'ee}}
				\fancyfoot[L]{ }
				\fancyfoot[C]{}
				\fancyfoot[R]{\scriptsize{\textcolor{gray}{version 2014-2015 (\today)}}}
				\pagestyle{plain}
				\reversemarginpar
				\usepackage{rotating}						
				\begin{document}
					\begin{textblock}{9}(2,3.2)
						\includegraphics[width=2cm]{../../../_templates/java/icons/logo-esi}
					\end{textblock}
				
				
				
				
				%\maketitle
				\cadretitre{TD1}{TD S\'equentiel - Rappels de base}
				\thispagestyle{fancy}
        \marginpar{\begin{sideways}
            \begin{minipage}[t]{1cm}
            \begin{tiny}
            \includegraphics[width=1\linewidth,height=1\textheight,keepaspectratio=true]{../../../_templates/java/icons/cc-gris.jpg}
			\end{tiny}
			\end{minipage}
            \begin{minipage}[b]{19cm}
            \begin{tiny}
            \textcolor{gray}{Distribué sous licence Creative Commons Paternité - Partage à l'Identique 2.0 Belgique 
            (\texttt{http://creativecommons.org/licenses/by-sa/2.0/be/})
			\vspace{-1em}
			\\Les autorisations au-delà du champ de cette licence peuvent être obtenues à 
			\texttt{http://www.heb.be/esi}
			- \texttt{mcodutti@heb.be}
			}\end{tiny}
			\end{minipage}
        \end{sideways}}
            \begin{abstract}
			Ces exercices ont pour but de v\'erifier que vous avez fix\'e les bases du LDA et de la programmation Java.
		
            \par
        \end{abstract}
				\vspace{-2em}\tableofcontents
				\pagestyle{plain}
            \clearpage
            \fancyhead[L,C,R]{}
            \fancyfoot[L,C]{}
            \fancyfoot[R]{ \scriptsize{\textcolor{gray}{
				InitSeq - page \thepage}}}
				\thispagestyle{fancy}
				\pagestyle{fancy}
	   
            \section{Variables et types}\subsection{Les variables et les types}
			
		\subparagraph{Le type des donn\'ees} 
		
                \textcolor{white}{.} \par
             
								Comment d\'eclarer en algorithmique :
							
					\begin{itemize}
				
			\item un montant d'un ticket de caisse ?  \textcolor{gray}{\underline{\hspace*{10em}}} 
			\item une cote sur le bulletin (pas de demi-point autoris\'e) ?  \textcolor{gray}{\underline{\hspace*{10em}}} 
			\item le titre d'un film ? \textcolor{gray}{\underline{\hspace*{10em}}} 
			\item l'initiale de votre nom ? \textcolor{gray}{\underline{\hspace*{10em}}} 
					\end{itemize}
				
			
		\subparagraph{Le type des donn\'ees} 
		
                \textcolor{white}{.} \par
             
								Comment d\'eclarer en Java :
							
					\begin{itemize}
				
			\item un montant d'un ticket de caisse ?  \textcolor{gray}{\underline{\hspace*{10em}}} 
			\item une cote sur le bulletin (pas de demi-point autoris\'e) ?  \textcolor{gray}{\underline{\hspace*{10em}}} 
			\item le titre d'un film ? \textcolor{gray}{\underline{\hspace*{10em}}} 
			\item l'initiale de votre nom ? \textcolor{gray}{\underline{\hspace*{10em}}} 
					\end{itemize}
				
			
		\subparagraph{Types Java} 
		
                \textcolor{white}{.} \par
            
                  Donnez l'\'equivalent Java des types vus en algorithmique. 
                
            \par
         
                entier 		 \textcolor{gray}{\underline{\hspace*{2em}}} \par
				
                r\'eel   		 \textcolor{gray}{\underline{\hspace*{5em}}} \par
				
                chaine 		 \textcolor{gray}{\underline{\hspace*{5em}}} \par
				
                caract\`ere	 \textcolor{gray}{\underline{\hspace*{3em}}} \par
				
                bool\'een      \textcolor{gray}{\underline{\hspace*{5em}}} \par
				
			
		\subparagraph{Conventions de nom} 
		
                \textcolor{white}{.} \par
            Cochez les noms qui respectent les conventions (sachant que nous travaillons dans le cadre de la tva)
            \begin{itemize} 
        
            \item[ \ding{"6F} ] \verb@double Taxe;@
        
            \item[ \ding{"6F} ] \verb|final double TVA;|
        
            \item[ \ding{"6F} ] \verb@double taxeValeurAjoutée;@
        
            \item[ \ding{"6F} ] \verb@final double TAUXTVA;@
        
            \item[ \ding{"6F} ] \verb|final double TAUX_TVA|
        
            \item[ \ding{"6F} ] \verb@double t;@
        
            \end{itemize} 
        
			
		\subparagraph{final} 
		
					\textcolor{white}{.} \par
				
            \par
        
					Soit le code :
				
            \par
        \begin{Java}
	public class Final {  
		public static void main (String[] args) {
			final double tva = 21.5; 
			System.out.println ((1000/100)*tva);  
			tva = 19;
		}
	}			\end{Java}
					qui, \`a la compilation, g\'en\`ere le message d'erreur 
				
            \par
        \begin{verbatim}
	Final.java:5:: cannot assign a value to final variable tva\end{verbatim}ce message veut dire que
            \begin{itemize} 
        
            \item[ \ding{"6D} ] le nom tva ne respecte pas la convention de nom
        
            \item[ \ding{"6D} ] final n'est pas un mot reconnu
        
            \item[ \ding{"6D} ] on ne peut assigner une deuxi\`eme fois une valeur \`a une variable final
        
            \item[ \ding{"6D} ] on ne peut assigner une valeur enti\`ere \`a tva 
        
            \end{itemize} 
        Si vous n'avez pas r\'epondu correctement \`a toutes les questions, au nom de variable pr\`es,
				r\'evisez ici (\url{www.heb.be/esi/InitSeq/fr/../../TDSeq/fr/html/unit\_VariablesEtTypes.html})
            \par
        \subsection{Java est un langage fortement typ\'e.}
			
		\subparagraph{Le type des donn\'ees} 
		
                \textcolor{white}{.} \par
             
								Toute donn\'ee a un type. Il existe 3 grands types de donn\'ees. Lesquels ?
							
					\begin{itemize}
				
			\item  \textcolor{gray}{\underline{\hspace*{16em}}} 
			\item  \textcolor{gray}{\underline{\hspace*{16em}}} 
			\item  \textcolor{gray}{\underline{\hspace*{20em}}} 
					\end{itemize}
				Si vous n'avez pas r\'epondu correctement \`a toutes les questions, 
				    r\'evisez ici (\url{www.heb.be/esi/InitSeq/fr/../../TDSeq/fr/html/VariablesEtTypes\_learningObject3.html})
            \par
        
			
		\subparagraph{Les types primitifs num\'eriques entiers} 
		
                \textcolor{white}{.} \par
             
								Donnez les types primitifs num\'eriques entiers dans l'ordre (de celui quiprend le moins de place en m\'emoire \`a celui qui en prend le plus) :
							
					\begin{itemize}
				
			\item  \textcolor{gray}{\underline{\hspace*{3em}}} 
			\item  \textcolor{gray}{\underline{\hspace*{3em}}} 
			\item  \textcolor{gray}{\underline{\hspace*{3em}}} 
			\item  \textcolor{gray}{\underline{\hspace*{2em}}} 
			\item  \textcolor{gray}{\underline{\hspace*{3em}}} 
					\end{itemize}
				Si vous n'avez pas r\'epondu correctement \`a toutes les questions, 
				    r\'evisez ici (\url{www.heb.be/esi/InitSeq/fr/../../TDSeq/fr/html/VariablesEtTypes\_learningObject3.html})
            \par
        
			
		\subparagraph{Les types primitifs num\'eriques flottants} 
		
                \textcolor{white}{.} \par
             
								Donnez les types primitifs num\'eriques flottants dans l'ordre (de celui quiprend le moins de place en m\'emoire \`a celui qui en prend le plus) :
							
					\begin{itemize}
				
			\item  \textcolor{gray}{\underline{\hspace*{3em}}} 
			\item  \textcolor{gray}{\underline{\hspace*{5em}}} 
					\end{itemize}
				Si vous n'avez pas r\'epondu correctement \`a toutes les questions, 
				    r\'evisez ici (\url{www.heb.be/esi/InitSeq/fr/../../TDSeq/fr/html/VariablesEtTypes\_learningObject3.html})
            \par
        
			
		\subparagraph{Le type primitif non num\'erique} 
		
                \textcolor{white}{.} \par
             
								Donnez le type primitif non num\'eriques :
							
					\begin{itemize}
				
			\item  \textcolor{gray}{\underline{\hspace*{5em}}} 
					\end{itemize}
				Si vous n'avez pas r\'epondu correctement \`a toutes les questions, 
          r\'evisez ici (\url{www.heb.be/esi/InitSeq/fr/../../TDSeq/fr/html/VariablesEtTypes\_learningObject3.html})
            \par
        \section{Op\'erateurs et expressions}\subsection{Op\'erateurs et expressions}
			
		\subparagraph{Op\'erateurs Java} 
		
                \textcolor{white}{.} \par
            
                  Donnez l'\'equivalent Java des op\'erateurs vu en algorithmique.
                
            \par
         
                *   \textcolor{gray}{\underline{\hspace*{1em}}} \par
				
                /   \textcolor{gray}{\underline{\hspace*{1em}}}  entre 2 r\'eels \par
				
                DIV   \textcolor{gray}{\underline{\hspace*{1em}}}  entre 2 entiers \par
				
                MOD   \textcolor{gray}{\underline{\hspace*{1em}}} \par
				
			
		\subparagraph{Que vaut ?} 
		
                \textcolor{white}{.} \par
            Que vaut l'expression en algo
                5 DIV 3 :  \textcolor{gray}{\underline{\hspace*{1em}}} 
			
		\subparagraph{Que vaut ?} 
		
                \textcolor{white}{.} \par
            Que vaut l'expression en algo
                5 MOD 3 :  \textcolor{gray}{\underline{\hspace*{1em}}} 
			
		\subparagraph{Que vaut ?} 
		
                \textcolor{white}{.} \par
            Que vaut l'expression java
                11/2 :  \textcolor{gray}{\underline{\hspace*{1em}}} 
			
		\subparagraph{Que vaut ?} 
		
                \textcolor{white}{.} \par
            Que vaut l'expression java
                11.0/2 :  \textcolor{gray}{\underline{\hspace*{2em}}} 
			
		\subparagraph{Que vaut ?} 
		
                \textcolor{white}{.} \par
            Que vaut l'expression java
                11\%3 :  \textcolor{gray}{\underline{\hspace*{1em}}} Si vous n'avez pas r\'epondu correctement \`a toutes les questions, 
          r\'evisez ici (\url{www.heb.be/esi/InitSeq/fr/../../TDSeq/fr/html/unit\_Op\'erateursEtExpressions.html})
            \par
        
			
		\subparagraph{Comment calculer ?} 
		
                \textcolor{white}{.} \par
            Comment calculer en Java
                la racine carr\'ee d'une variable nomm\'ee nb ? :  \textcolor{gray}{\underline{\hspace*{10em}}} Si vous n'avez pas r\'epondu correctement \`a la question, 
          r\'evisez ici (\url{www.heb.be/esi/InitSeq/fr/../../TDSeq/fr/html/Op\'erateursEtExpressions\_learningObject6.html})
            \par
        \section{L'affectation d'une valeur \`a une variable}
				Cette op\'eration est probablement l'op\'eration la plus importante. En effet, une variable ne
        prend son sens r\'eel que si elle re\c coit \`a un moment donn\'e une valeur. Il y a deux moyens de
        donner une valeur \`a une variable.
      
            \par
        \subsection{Affectation interne en algo}
			
		\subparagraph{} 
		
                \textcolor{white}{.} \par
            Les exemples d'affectation sont-ils corrects ?
						
            \begin{itemize} 
        
            \item[ \ding{"6F} ] \begin{verbatim}

somme, nombre1, nombre2 : entiers 
nombre1 ← 3 
nombre2 ← 5 
somme ← nombre1 + nombre2\end{verbatim}
        
            \item[ \ding{"6F} ] \begin{verbatim}

denRes, den1, den2 : réels 
den1 ← 3.5 
den2 ← 1 
denRes ← den1 * den2\end{verbatim}
        
            \item[ \ding{"6F} ] \begin{verbatim}
						
cpt : entier 
cpt ← cpt + 1\end{verbatim}
        
            \item[ \ding{"6F} ] \begin{verbatim}
						
delta, a, b, c : réels 
a ← 4 
b ← 3/2 
c ← 7/4 
delta ← b**2 – 4*a*c\end{verbatim}
        
            \item[ \ding{"6F} ] \begin{verbatim}
						
maChaine : chaine 
maChaine ← "Bonjour"\end{verbatim}
        
            \item[ \ding{"6F} ] \begin{verbatim}
						
a, b : entier 
test : booléen 
a ← 4 
b ← 3 
test ← a = b\end{verbatim}
        
            \item[ \ding{"6F} ] \begin{verbatim}
						
somme : entier 
somme + 1 ← 3\end{verbatim}
        
            \item[ \ding{"6F} ] \begin{verbatim}
						
somme, n : entier 
n ← 4 
somme ← 3n\end{verbatim}
        
            \end{itemize} 
        Si vous n'avez pas r\'epondu correctement \`a toutes les questions, 
        r\'evisez ici (\url{www.heb.be/esi/InitSeq/fr/../../TDSeq/fr/html/Affectation\_learningObject1.html})
            \par
        \subsection{Affectation interne en Java}
			
		\subparagraph{} 
		
                \textcolor{white}{.} \par
            Les exemples d'affectation sont-ils corrects ?
						
            \begin{itemize} 
        
            \item[ \ding{"6F} ] \begin{Java}
  
int somme;
int nombre1;
int nombre 2;
nombre1 = 3;
nombre2 = -8;
somme = nombre1 + nombre2;\end{Java}
        
            \item[ \ding{"6F} ] \begin{Java}

double denRes;
double den1;
double den2;
den1 = 3.5;
den2 = 1;
denRes = den1 * den2;\end{Java}
        
            \item[ \ding{"6F} ] \begin{Java}
						
int cpt; 
cpt = cpt + 1;\end{Java}
        
            \item[ \ding{"6F} ] \begin{Java}
						
double delta; 
double a; 
double b; 
double c; 
a = 4; 
b = 3.0/2; 
c = 7.0/4; 
delta = b*b - 4*a*c;\end{Java}
        
            \item[ \ding{"6F} ] \begin{Java}
						
String maChaine; 
maChaine = "Bonjour";\end{Java}
        
            \item[ \ding{"6F} ] \begin{Java}
						
int a; 
int b; 
boolean test;
a = 4;
b = 3;
test = a = b;\end{Java}
        
            \item[ \ding{"6F} ] \begin{Java}
						
int somme; 
somme + 1 = 3;\end{Java}
        
            \item[ \ding{"6F} ] \begin{Java}
						
int somme;
int n;
n = 4;
somme = 3n;\end{Java}
        
            \end{itemize} 
        Si vous n'avez pas r\'epondu correctement \`a toutes les questions, 
        r\'evisez ici (\url{www.heb.be/esi/InitSeq/fr/../../TDSeq/fr/html/Affectation\_learningObject2.html})
            \par
        \subsection{Affectation externe en algo}
			
		\subparagraph{} 
		
                \textcolor{white}{.} \par
            Les exemples d'affectation sont-ils corrects ?
						
            \begin{itemize} 
        
            \item[ \ding{"6F} ] nombre1, nombre2 : entiers \par
				
						lire nombre1, nombre2
        
            \item[ \ding{"6F} ] nombre1 : entier \par
				
						lire nombre1
        
            \item[ \ding{"6F} ] maChaine : chaine \par
				
						lire maChaine
        
            \item[ \ding{"6F} ] maChaine : chaine \par
				
						lire "Entrez une chaine", maChaine
        
            \item[ \ding{"6F} ] lire nombre1 + nombre2
        
            \end{itemize} 
        Si vous n'avez pas r\'epondu correctement \`a toutes les questions, 
        r\'evisez ici (\url{www.heb.be/esi/InitSeq/fr/../../TDSeq/fr/html/Affectation\_learningObject3.html})
            \par
        \subsection{Affectation externe en Java}
			
		\subparagraph{S\'election multiple} 
		
                \textcolor{white}{.} \par
            Quels morceaux de code, parmi les suivants, NE sont PAS corrects ? Pourquoi ?
						
            \begin{itemize} 
        
            \item[ \ding{"6F} ]  
							code 1
							\begin{Java}
    public class Exercice { 
        public static void main(String[] args) { 
            Scanner clavier = new Scanner(System.in); 
            int nombre1; 
            nombre1 = clavier.nextInt();
        } 
    }						\end{Java}
        
            \item[ \ding{"6F} ]  
							code 2
							\begin{Java}
    import java.util.Scanner;
    public class Exercice {  
        public static void main(String[] args) {
            Scanner clavier = new Scanner(System.in);
            int nombre1;
            nombre1 = clavier.nextInt();
        }  
    }						\end{Java}
        
            \item[ \ding{"6F} ]  
							code 3
							\begin{Java}
    import java.util.Scanner;
    public class Exercice {
        public static void main(String[] args ) {
            Scanner clavier = new Scanner(System.in); 
            double nombre1;  
            nombre1 = clavier.nextDouble();  
        }
    }						\end{Java}
        
            \item[ \ding{"6F} ]  
							code 4
							\begin{Java}
    import java.util.Scanner;
    public class Exercice { 
        public static void main(String[] args) {
            int nombre1;  
            nombre1 = clavier.nextInt(); 
        }
    }						\end{Java}
        
            \end{itemize} 
        
			
		\subparagraph{M\'ethodes de Scanner} 
		
                \textcolor{white}{.} \par
            Compl\'etez les m\'ethodes suivantes afin de permettre une lecture au clavier ad\'equate.\begin{Java}
import java.util.Scanner;
public class ReadingExercice {
    public static void main (String[] args) {  
        Scanner keyboard = new Scanner(System.in);  
        int integerNb = keyboard. \\_\\_\\_\\_\\_\\_\\_\\_\\_\\_\\_\\_\\_\\_\\_\\_ ;
        double realNb = keyboard. \\_\\_\\_\\_\\_\\_\\_\\_\\_\\_\\_\\_\\_\\_\\_\\_  ;
        boolean ok = keyboard. \\_\\_\\_\\_\\_\\_\\_\\_\\_\\_\\_\\_\\_\\_\\_\\_  ;
        String string1 = keyboard. \\_\\_\\_\\_\\_\\_\\_\\_  ; // We read a word
        String string2 = keyboard. \\_\\_\\_\\_\\_\\_\\_\\_\\_\\_\\_\\_\\_\\_\\_\\_  ; // We read a line 
        char aCharacter = keyboard. \\_\\_\\_\\_\\_\\_\\_\\_\\_\\_\\_\\_\\_\\_\\_\\_  ;
    }
}							\end{Java}Si vous n'avez pas r\'epondu correctement \`a toutes les questions, 
        r\'evisez ici (\url{www.heb.be/esi/InitSeq/fr/../../TDSeq/fr/html/Affectation\_learningObject4.html})
            \par
        \subsection{Communication des r\'esultats en algo}
			
		\subparagraph{} 
		
                \textcolor{white}{.} \par
            Les exemples de communication des r\'esultats sont-ils corrects ?
						
            \begin{itemize} 
        
            \item[ \ding{"6F} ] \begin{verbatim}

somme, nombre1, nombre2 : entiers 
nombre1 ← 3 
nombre2 ← 5 
somme ← nombre1 + nombre2
afficher somme\end{verbatim}
        
            \item[ \ding{"6F} ] \begin{verbatim}

denRes, den1, den2 : réels 
den1 ← 3.5 
den2 ← 1 
denRes ← den1 * den2
afficher den1, "/", den2, " ", denRes \end{verbatim}
        
            \item[ \ding{"6F} ] \begin{verbatim}
						
cpt : entier 
afficher cpt\end{verbatim}
        
            \item[ \ding{"6F} ] \begin{verbatim}
						
a, b, c : réels 
a ← 4 
b ← 3/2 
c ← 7/4 
afficher b**2 – 4*a*c\end{verbatim}
        
            \item[ \ding{"6F} ] \begin{verbatim}
						
afficher "Bonjour"\end{verbatim}
        
            \item[ \ding{"6F} ] \begin{verbatim}
						
a, b : entier 
test : booléen 
a ← 4 
b ← 3 
afficher a = b\end{verbatim}
        
            \end{itemize} 
        Si vous n'avez pas r\'epondu correctement \`a toutes les questions, 
        r\'evisez ici (\url{www.heb.be/esi/InitSeq/fr/../../TDSeq/fr/html/Affectation\_learningObject5.html})
            \par
        \subsection{Communication des r\'esultats en Java}
			
		\subparagraph{S\'election multiple} 
		
                \textcolor{white}{.} \par
            Quels morceaux de code, parmi les suivants, sont corrects ? Pourquoi ?
						
            \begin{itemize} 
        
            \item[ \ding{"6F} ]  
							code 1
							\begin{Java}
    import java.util.Scanner;
    public class Exercice { 
        public static void main(String[] args) { 
            Scanner clavier = new Scanner(System.in); 
            int nombre1; 
            nombre1 = clavier.nextInt();
            System.out.println(nombre1);
        } 
    }						\end{Java}
        
            \item[ \ding{"6F} ]  
							code 2
							\begin{Java}
    import java.util.Scanner;
    public class Exercice {  
        public static void main(String[] args) {
            Scanner clavier = new Scanner(System.in);
            int nombre1;
            int nombre2;
            nombre1 = clavier.nextInt();
            nombre2 = clavier.nextInt();
            System.out.println(nombre1 + nombre2);
        }  
    }						\end{Java}
        
            \item[ \ding{"6F} ]  
							code 3
							\begin{Java}
    import java.util.Scanner;
    public class Exercice {  
        public static void main(String[] args) {
            Scanner clavier = new Scanner(System.in);
            int nombre1;
            int nombre2;
            nombre1 = clavier.nextInt();
            nombre2 = clavier.nextInt();
            System.out.println(nombre1 + " " + nombre2);
        }  
    }						\end{Java}
        
            \item[ \ding{"6F} ]  
							code 4
							\begin{Java}
    import java.util.Scanner;
    public class Exercice {  
        public static void main(String[] args) {
            Scanner clavier = new Scanner(System.in);
            int nombre1;
            int nombre2;
            nombre1 = clavier.nextInt();
            nombre2 = clavier.nextInt();
            System.out.println(nombre1, nombre2);
        }  
    }						\end{Java}
        
            \end{itemize} 
        
			
		\subparagraph{M\'ethodes de Scanner} 
		
                \textcolor{white}{.} \par
            Compl\'etez les m\'ethodes suivantes afin de permettre une lecture au clavier ad\'equate.\begin{Java}
import java.util.Scanner;
public class ReadingExercice {
    public static void main (String[] args) {  
        Scanner keyboard = new Scanner(System.in);  
        int integerNb = keyboard. \\_\\_\\_\\_\\_\\_\\_\\_\\_\\_\\_\\_\\_\\_\\_\\_ ;
        double realNb = keyboard. \\_\\_\\_\\_\\_\\_\\_\\_\\_\\_\\_\\_\\_\\_\\_\\_  ;
        boolean ok = keyboard. \\_\\_\\_\\_\\_\\_\\_\\_\\_\\_\\_\\_\\_\\_\\_\\_  ;
        String string1 = keyboard. \\_\\_\\_\\_\\_\\_\\_\\_  ; // We read a word
        String string2 = keyboard. \\_\\_\\_\\_\\_\\_\\_\\_\\_\\_\\_\\_\\_\\_\\_\\_  ; // We read a line 
        char aCharacter = keyboard. \\_\\_\\_\\_\\_\\_\\_\\_\\_\\_\\_\\_\\_\\_\\_\\_  ;
    }
}							\end{Java}Si vous n'avez pas r\'epondu correctement \`a toutes les questions, 
        r\'evisez ici (\url{www.heb.be/esi/InitSeq/fr/../../TDSeq/fr/html/Affectation\_learningObject6.html})
            \par
        \section{Structure g\'en\'erale}\subsection{Structure g\'en\'erale d'un algorithme}
          Pour ces exercices, nous vous demandons de comprendre des algorithmes donn\'es. 
          
			
		\subparagraph{Compr\'ehension} 
		
                \textcolor{white}{.} \par
            
							  Que vont-ils afficher si \`a chaque fois les deux nombres lus au d\'epart sont successivement 2 et 3 ?
							
					\begin{itemize}
				
			\item \begin{verbatim}
module exerciceA()
  a,b : entiers
  quotient : réel
  lire b, a
  quotient ← a / b
  afficher quotient
fin module
				\end{verbatim} \textcolor{gray}{\underline{\hspace*{2em}}} 
			\item \begin{verbatim}
module exerciceB()
  a, b, c, d : entiers
  lire c, d
  a ← 2*c+5*d
  b ← 2+c*3+d
  c ← a MOD b
  afficher a DIV c
fin module
				\end{verbatim} \textcolor{gray}{\underline{\hspace*{1em}}} 
			\item \begin{verbatim}
module exerciceC()
  x, y : réels
  lire x, y
  x ← x*x 
  x ← x*x+y*y 
  x ← √x
  afficher x
fin module
				\end{verbatim} \textcolor{gray}{\underline{\hspace*{1em}}} 
			\item \begin{verbatim}
module exerciceD()
  x, y : réels
  lire x, x
  x ← x MOD x + (x + 1) DIV 2
  afficher x + 3
fin module
				\end{verbatim} \textcolor{gray}{\underline{\hspace*{1em}}} 
					\end{itemize}
				
            \par
        Si vous n'avez pas r\'epondu correctement \`a toutes les questions, 
        r\'evisez ici (\url{www.heb.be/esi/InitSeq/fr/../../TDSeq/fr/html/Structure\_learningObject1.html})
            \par
        \subsection{Structure g\'en\'erale d'un programme Java, comment le compiler et l'ex\'ecuter.}
			
		\subparagraph{Structure d'un programme} 
		
                \textcolor{white}{.} \par
            Quelles structures g\'en\'erales d'un programme sont correctes 
							(c-\`a-d qu'il doit compiler et respecter les conventions) parmi les suivantes ?
						
            \begin{itemize} 
        
            \item[ \ding{"6F} ]  
							code 1
							\begin{verbatim}
    public class exercice {
        // put methods here  
    }\end{verbatim}
        
            \item[ \ding{"6F} ]  
							code 2
							\begin{verbatim}
    public CLASS Exercice {
        // put methods here  
    }\end{verbatim}
        
            \item[ \ding{"6F} ] 
							code 3 
							\begin{verbatim}
    public class Exercice {  
        // put methods here  
    }\end{verbatim}
        
            \item[ \ding{"6F} ]  
							code 4 
							\begin{verbatim}
    public class MonExercice {  
        // put methods here  
    }\end{verbatim}
        
            \item[ \ding{"6F} ]  
							code 5 
							\begin{verbatim}
    PUBLIC CLASS EXERCICE { 
        // put methods here  
    }\end{verbatim}
        
            \end{itemize} 
        
			
		\subparagraph{Nom d'un programme} 
		
                \textcolor{white}{.} \par
            Quels noms doivent avoir les fichiers dans lesquels sont plac\'es les programmes suivants :\begin{Java}
    public class Exercice {  
        // Methods    
    }						\end{Java} \textcolor{gray}{\underline{\hspace*{10em}}} \begin{Java}
    public class SommeChiffres {
        // Methods    
    }						\end{Java} \textcolor{gray}{\underline{\hspace*{16em}}} \begin{Java}
    public class sommechiffres {    
        // Methods    
    }						\end{Java} \textcolor{gray}{\underline{\hspace*{16em}}} 
			
		\subparagraph{Compiler/ex\'ecuter} 
		
                \textcolor{white}{.} \par
             
							Quelle commande permet de compiler le fichier nomm\'e \verb@SommeChiffres.java@ ?  
							\par
				 \textcolor{gray}{\underline{\hspace*{3em}}}  \textcolor{gray}{\underline{\hspace*{16em}}} \par
				
							Quelle commande permet d'ex\'ecuter ce programme ?  
							\par
				 \textcolor{gray}{\underline{\hspace*{3em}}}  \textcolor{gray}{\underline{\hspace*{10em}}} 
			
		\subparagraph{M\'ethode principale} 
		
                \textcolor{white}{.} \par
            
								Comment s'\'ecrit l'ent\^ete de la m\'ethode principale (1 mot par case) ?
							
            \par
         \textcolor{gray}{\underline{\hspace*{5em}}}  \textcolor{gray}{\underline{\hspace*{5em}}}  \textcolor{gray}{\underline{\hspace*{3em}}}  \textcolor{gray}{\underline{\hspace*{3em}}}  
							(            
							 \textcolor{gray}{\underline{\hspace*{5em}}}  \textcolor{gray}{\underline{\hspace*{2em}}}  \textcolor{gray}{\underline{\hspace*{3em}}}  
							)   
						
			
		\subparagraph{Compr\'ehension} 
		
                \textcolor{white}{.} \par
            
							  Que vont afficher les codes java suivants si \`a chaque fois les deux nombres lus au d\'epart sont successivement 2 et 3 ?
							
					\begin{itemize}
				
			\item \begin{Java}
import java.util.Scanner;
public class Exercice1 {
    public static void main(String [] args) {
        Scanner clavier = new Scanner(System.in);
        int nb1 = clavier.nextInt();
        int nb2 = clavier.nextInt();
        System.out.println(nb1 + " " + nb2);
    }
}
        \end{Java} \textcolor{gray}{\underline{\hspace*{2em}}} 
			\item \begin{Java}
import java.util.Scanner;
public class Exercice2 {
    public static void main(String [] args) {
        Scanner clavier = new Scanner(System.in);
        int nb1 = clavier.nextInt();
        int nb2 = clavier.nextInt();
        int nb3 = 2*nb1 + nb2;
        System.out.println(nb3);
    }
}
									\end{Java} \textcolor{gray}{\underline{\hspace*{1em}}} 
			\item \begin{Java}
import java.util.Scanner;
public class Exercice3 {
    public static void main(String [] args) {
        Scanner clavier = new Scanner(System.in);
        int nb1 = clavier.nextInt();
        int nb2 = clavier.nextInt();
        System.out.println(nb2/nb1);
    }
}
        \end{Java} \textcolor{gray}{\underline{\hspace*{1em}}} 
			\item \begin{Java}
import java.util.Scanner;
public class Exercice4 {
    public static void main(String [] args) {
        Scanner clavier = new Scanner(System.in);
        int nb1 = clavier.nextInt();
        int nb2 = clavier.nextInt();
        System.out.println(nb1%nb2);
    }
}
        \end{Java} \textcolor{gray}{\underline{\hspace*{1em}}} 
			\item \begin{Java}
import java.util.Scanner;
public class Exercice5 {
    public static void main(String [] args) {
        Scanner clavier = new Scanner(System.in);
        int nb1 = clavier.nextInt();
        nb1 = clavier.nextInt();
        nb1 = nb1 * nb1;
        System.out.println(Math.sqrt(nb1));
    }
}
        \end{Java} \textcolor{gray}{\underline{\hspace*{1em}}} 
					\end{itemize}
				Si vous n'avez pas r\'epondu correctement \`a toutes les questions, 
        r\'evisez ici (\url{www.heb.be/esi/InitSeq/fr/../../TDSeq/fr/html/Structure\_learningObject2.html})
            \par
        \section{Exercices complets}\subsection{\`A vous de jouer...}
			
		\subparagraph{Consignes} 
		
					\textcolor{white}{.} \par
				
          Il est temps de se lancer et d'\'ecrire vos premiers modules et programmes Java correspondant. 
          Voici quelques conseils pour vous guider dans la r\'esolution de tels probl\`emes :
          
					\begin{itemize}
				
			\item il convient d'abord de bien comprendre le probl\`eme pos\'e ; assurez-vous qu'il est parfaitement sp\'ecifi\'e ;
			\item mettez en \'evidence les variables \textbf{\guillemotleft  donn\'ees \guillemotright }, les variables \textbf{\guillemotleft  r\'esultats \guillemotright } et les variables de travail ;
			\item n'h\'esitez pas \`a faire une \'ebauche de r\'esolution en fran\c cais avant d'\'elaborer l'algorithme d\'efinitif pseudo-cod\'e ;
			\item d\'eclarez ensuite les variables (et leur type) qui interviennent dans l'algorithme ; les noms des variables risquant de ne pas \^etre suffisamment explicites.
			\item \'Ecrivez la partie algorithmique \textbf{AVANT} de vous lancer dans la programmation en Java.
					\end{itemize}
				
            \par
        
			
		\subparagraph{Exercices} 
		
					\textcolor{white}{.} \par
				
        \'Ecrivez les algorithmes et codez les programmes Java correspondant qui 
          
					\begin{enumerate}
				
			\item r\'ealise la permutation du contenu de deux variables.
			\item \'etant donn\'e un moment dans la journ\'ee donn\'e par trois nombres lus, \`a savoir, heure, minute et seconde, calcule le nombre de secondes \'ecoul\'ees depuis minuit.
			\item \'etant donn\'e un temps \'ecoul\'e dans la journ\'ee exprim\'e en secondes, calcule et affiche ce temps sous la forme de trois nombres (heure, minute et seconde). \par
				
            Ex : 10000 secondes donnera 2h 46'40”
					\end{enumerate}
				
            \par
        
          \'Ecrivez un algorithme qui lit un nombre pouvant prendre des valeurs de 100 \`a 999 et qui affiche ce m\^eme nombre mais renvers\'e. 
          Le dernier chiffre devient le premier et inversement. \par
				
          Par exemple : le nombre 123 devient 321 et le nombre 410 deviendra 14.
        
            \par
        Pour plus d'exercices, 
        r\'evisez ici (\url{www.heb.be/esi/InitSeq/fr/../../TDSeq/fr/html/unit\_Exercices.html})
            \par
        
				\end{document}
			