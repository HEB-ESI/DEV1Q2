\documentclass[11pt,a4paper]{article}
			\usepackage[french]{babel}
					
				\usepackage{pifont}  
				\usepackage[utf8x]{inputenc}
				\usepackage[T1]{fontenc} 
				\usepackage{lmodern}			
				\usepackage{fancyhdr}
				\usepackage{textcomp}
				\usepackage{makeidx}
				\usepackage{tabularx}
				\usepackage{multicol}
				\usepackage{multirow}
				\usepackage{longtable}
				\usepackage{color}
				\usepackage{soul}
				\usepackage{boxedminipage}
				\usepackage{shadow}
				\usepackage{framed}			
				\usepackage{array}
				\usepackage{url}
				\usepackage{ragged2e}
				\usepackage{fancybox}
				\newcommand{\cadretitre}[2]{
				  \vspace*{0.8\baselineskip}
				  \begin{center}%
				  \boxput*(0,1){%
					%\colorbox{white}{\Large\textbf{\ #1\ }}%
				  }%
				  {%
					\setlength{\fboxsep}{10pt}%
				    \Ovalbox{\begin{minipage}{.8\linewidth}\begin{center}\Large\sffamily{#2}\end{center}\end{minipage}}}%
				  \end{center}
				  \vspace*{2\baselineskip}
				  }
			
			\makeatletter
			\def\@seccntformat#1{\protect\makebox[0pt][r]{\csname the#1\endcsname\quad}}
			\makeatother

				% Permet d'afficher qqchose à une positin absolue
				\usepackage[absolute]{textpos}
				\setlength{\TPHorizModule}{1cm}
				\setlength{\TPVertModule}{\TPHorizModule}
	
				\usepackage[titles]{tocloft}
				\setlength{\cftbeforesecskip}{0.5ex}
				\setlength{\cftbeforesubsecskip}{0.2ex}
				\addto\captionsfrench{\renewcommand\contentsname{}}
				
				\usepackage[font=scriptsize]{caption}
				
				\usepackage{listings}
\lstdefinestyle{lstverb}
  {
    basicstyle=\footnotesize,
    frameround=tttt, frame=trbl, framerule=0pt, rulecolor=\color{gray},
    lineskip=-1pt,   % pour rapprocher les lignes
    flexiblecolumns, escapechar=\\,
    tabsize=4, extendedchars=true
  }
\lstnewenvironment{Java}[1][]{\lstset{style=lstverb,language=java,#1}}{}
				\ifx\pdfoutput\undefined
					\usepackage{graphicx}
				\else
					\usepackage[pdftex]{graphicx}
				\fi
				\usepackage[a4paper, hyperfigures=true, colorlinks, linkcolor=black, citecolor=blue,urlcolor=blue, pagebackref=true, bookmarks=true, bookmarksopen=true,bookmarksnumbered=true,
                pdfauthor={}, pdftitle={TD Alternatives}, pdfkeywords={TD Alternatives, },pdfpagemode=UseOutlines,pdfpagetransition=Dissolve,nesting=true,
				backref, pdffitwindow=true, bookmarksnumbered=true]{hyperref}
				\usepackage{supertabular}
				\usepackage[table]{xcolor}
				\usepackage{url}
				\usepackage{caption} 
				\setlength{\parskip}{1.3ex plus 0.2ex minus 0.2ex}
				\setlength{\parindent}{0pt}
				
				\makeatletter
				\def\url@leostyle{ \@ifundefined{selectfont}{\def\UrlFont{\sf}}{\def\UrlFont{\footnotesize\ttfamily}}}
				\makeatother
				\urlstyle{leo}
				
				\definecolor{examplecolor}{rgb}{0.156,0.333,0.443}
				\definecolor{definitioncolor}{rgb}{0.709,0.784,0.454}
				\definecolor{exercisecolor}{rgb}{0.49,0.639,0}
				\definecolor{hintcolor}{rgb}{0.941,0.674,0.196}
				\definecolor{tableHeadercolor}{rgb}{0.709,0.784,0.454}
				\definecolor{tablerowAltcolor}{rgb}{.866,.905,.737}
				\definecolor{tablerowAlt2color}{rgb}{.968,.976,.933}
				\definecolor{verylightgray}{rgb}{0.98,0.98,0.98}
				
				\newenvironment{fshaded}{
				\def\FrameCommand{\fcolorbox{framecolor}{shadecolor}}
				\MakeFramed {\FrameRestore}}
				{\endMakeFramed}
				
				\newenvironment{fexample}[1][]{\definecolor{shadecolor}{rgb}{.913,.913,.913}
				\definecolor{framecolor}{rgb}{.156,.333,.443}
				\begin{fshaded}}{\end{fshaded}} 
				
				\newenvironment{fdefinition}{\definecolor{shadecolor}{rgb}{.913,.913,.913}
				\definecolor{framecolor}{rgb}{.709,.784,.454}
				\begin{fshaded}}{\end{fshaded}}
				
				\newenvironment{fexercise}{\definecolor{shadecolor}{rgb}{.913,.913,.913}
				\definecolor{framecolor}{rgb}{.49,.639,0}
				\begin{fshaded}}{\end{fshaded}}
				
				\newenvironment{fhint}{\definecolor{shadecolor}{rgb}{.913,.913,.913}
				\definecolor{framecolor}{rgb}{.941,.674,.196}
				\begin{fshaded}}{\end{fshaded}}	
				
				\newcommand{\PreserveBackslash}[1]{
				\let\temp=\\#1\let\\=\temp
				}
				\let\PBS=\PreserveBackslash
				\newcolumntype{A}{>{\PBS\raggedright\small\hspace{0pt}}X}
				\newcolumntype{L}[1]{>{\PBS\raggedright\small\hspace{0pt}}p{#1}}
				\newcolumntype{R}[1]{>{\PBS\raggedleft\small\hspace{0pt}}p{#1}}
				\newcolumntype{C}[1]{>{\PBS\centering\small\hspace{0pt}}p{#1}}
				
				\makeindex
				
				\title{TD Alternatives}	
			\date{}
			\author{\scriptsize{}}
			\definecolor{light-gray}{gray}{0.8}
			\renewcommand{\headrulewidth}{0pt}
			\fancyhead[L]{
				\footnotesize\textsc{Haute \'Ecole de Bruxelles}\\
	    			\footnotesize\textsc{\'Ecole Sup\'erieure d'Informatique}
			}
			\fancyhead[R]{
				\footnotesize{Bachelor en Informatique}\\
				\footnotesize{Laboratoires Java} - 
			\footnotesize{1\`ere ann\'ee}}
				\fancyfoot[L]{ }
				\fancyfoot[C]{}
				\fancyfoot[R]{\scriptsize{\textcolor{gray}{version 2014-2015 (\today)}}}
				\pagestyle{plain}
				\reversemarginpar
				\usepackage{rotating}						
				\begin{document}
					\begin{textblock}{9}(2,3.2)
						\includegraphics[width=2cm]{../../../_templates/java/icons/logo-esi}
					\end{textblock}
				
				
				
				
				%\maketitle
				\cadretitre{TD1}{TD Alternatives}
				\thispagestyle{fancy}
        \marginpar{\begin{sideways}
            \begin{minipage}[t]{1cm}
            \begin{tiny}
            \includegraphics[width=1\linewidth,height=1\textheight,keepaspectratio=true]{../../../_templates/java/icons/cc-gris.jpg}
			\end{tiny}
			\end{minipage}
            \begin{minipage}[b]{19cm}
            \begin{tiny}
            \textcolor{gray}{Distribué sous licence Creative Commons Paternité - Partage à l'Identique 2.0 Belgique 
            (\texttt{http://creativecommons.org/licenses/by-sa/2.0/be/})
			\vspace{-1em}
			\\Les autorisations au-delà du champ de cette licence peuvent être obtenues à 
			\texttt{http://www.heb.be/esi}
			- \texttt{mcodutti@heb.be}
			}\end{tiny}
			\end{minipage}
        \end{sideways}}
            \begin{abstract}
			Ce TD a pour but d'aborder les structures alternatives qui permettent de conditionner
      des parties d'algorithmes, de code. Elles ne seront ex\'ecut\'ees que si une condition est satisfaite.
		
            \par
        \end{abstract}
				\vspace{-2em}\tableofcontents
				\pagestyle{plain}
            \clearpage
            \fancyhead[L,C,R]{}
            \fancyfoot[L,C]{}
            \fancyfoot[R]{ \scriptsize{\textcolor{gray}{
				TDAlt - page \thepage}}}
				\thispagestyle{fancy}
				\pagestyle{fancy}
	   
            \section{si - alors - sinon}Cette structure permet d'ex\'ecuter une partie de code 
    ou une autre en fonction de la valeur de v\'erit\'e d'une condition.\subsection{si-alors-fin si}En pseudo-code :
            \par
        \begin{verbatim}
si condition alors
      // instructions à réaliser si la condition est VRAIE
fin si
      \end{verbatim}
        La \textbf{condition} est une expression d\'elivrant un r\'esultat 
        \textbf{bool\'een} (vrai ou faux) ; elle associe
        des variables, constantes, expressions arithm\'etiques, au moyen des op\'erateurs logiques ou de
        comparaison. En particulier, cette condition peut \^etre r\'eduite \`a une seule variable bool\'eenne.
      
            \par
        
        Dans cette structure, lorsque la \textbf{condition est vraie}, il y a ex\'ecution 
        de la s\'equence d'instructions contenue entre les mots \textbf{alors} 
        et \textbf{fin si} ; ensuite, l'algorithme continue de fa\c con
        s\'equentielle.
      
            \par
        
        Lorsque la \textbf{condition est fausse}, 
        les instructions se trouvant entre \textbf{alors} 
        et \textbf{fin si} sont tout simplement \textbf{ignor\'ees}.
      
            \par
        \begin{verbatim}
module Test
    nombre1 : entier
    lire nombre1 
    si (nombre1 < 0) alors
      afficher nombre1, " est négatif"
    fin si
fin module
    \end{verbatim}En Java :
            \par
        \begin{Java}
if (condition) { 
      // instructions si la condition est VRAIE
}
      \end{Java}
        La \textbf{condition} est une expression d\'elivrant un r\'esultat 
        \textbf{bool\'een} (true ou false).
      
            \par
        
        Dans cette structure, lorsque la \textbf{condition est vraie}, il y a ex\'ecution 
        de la s\'equence d'instructions contenue entre les mots \textbf{{} 
        et \textbf{}} ; ensuite, le code continue de fa\c con s\'equentielle.
      
            \par
        
        Lorsque la \textbf{condition est fausse}, 
        les instructions se trouvant entre \textbf{{} 
        et \textbf{}} sont tout simplement \textbf{ignor\'ees}.
      
            \par
        \begin{Java}
import java. util .Scanner;
public class Test {
    public static void main(String [] args) {
      Scanner clavier = new Scanner(System.in);
      int nombre1;
      nombre1 = clavier. nextInt ();
      if (nombre1 < 0) {
        System.out.println (nombre1 + " est negatif");
      }
    }
}
    \end{Java}\subsection{si-alors-sinon-fin si}En pseudo-code :
            \par
        \begin{verbatim}
si condition alors
      // instructions à réaliser si la condition est VRAIE
sinon
      // instructions à réaliser si la condition est FAUSSE
fin si
      \end{verbatim}
        Dans cette structure, une et une seule des deux s\'equences est ex\'ecut\'ee.
      
            \par
        \begin{verbatim}
// Lit un nombre et affiche si ce nombre est positif (zéro inclus) ou strictement négatif
module signeNombre()
    nb : entier
    lire nb
    si nb < 0 alors
      afficher "le nombre", nb, " est négatif"
    sinon
      afficher "le nombre", nb, " est positif ou nul"
    fin si
fin module
    \end{verbatim}En Java :
            \par
        \begin{Java}
if (condition) { 
      // instructions si la condition est VRAIE
} else { 
      // instructions si la condition est FAUSSE
} 
      \end{Java}\begin{Java}
import java. util .Scanner;
public class Test {
    public static void main(String [] args) {
      Scanner clavier = new Scanner(System.in);
      int nombre1;
      nombre1 = clavier. nextInt ();
      System.out. println (nombre1 + " est un nombre ");
      if (nombre1 < 0) {
        System.out. println (" negatif ");
      } else {
        System.out. println (" positif ");
      }
    }
}
    \end{Java}\section{selon que}Avec ces structures, plusieurs branches d'ex\'ecution sont disponibles. L'ordinateur choisit la
    branche \`a ex\'ecuter en fonction de la valeur d'une variable (ou parfois d'une expression) ou
    de la condition qui est vraie.\subsection{selon que (version avec listes de valeurs)}En pseudo-code :
            \par
        \begin{verbatim}
selon que variable vaut
    liste_1 de valeurs séparées par des virgules :
      // instructions lorsque la valeur de la variable est dans liste_1
    liste_2 de valeurs séparées par des virgules :
      // instructions lorsque la valeur de la variable est dans liste_2
    ...
    liste_n de valeurs séparées par des virgules :
      // instructions lorsque la valeur de la variable est dans liste_n
    autres :
      // instructions lorsque la valeur de la variable
      // ne se trouve dans aucune des listes précédentes
fin selon que
      \end{verbatim}Notez que le cas \verb@autres@ est facultatif.
            \par
        
        Dans ce type de structure, comme pour la structure \verb@si-alors-sinon@, une seule des s\'equences
        d'instructions sera ex\'ecut\'ee. On veillera \`a ne pas faire apparaitre une m\^eme valeur dans
        plusieurs listes. Cette structure est une simplification d'\'ecriture de plusieurs alternatives
        imbriqu\'ees.
      
            \par
        
        Elle est \'equivalente \`a : 
      
            \par
        \begin{verbatim}
si variable = une des valeurs de la liste_1 alors
    // instructions lorsque la valeur est dans liste_1
sinon
    si variable = une des valeurs de la liste_2 alors
      // instructions lorsque la valeur est dans liste_2
    sinon
      ...
      si variable = une des valeurs de la liste_n alors
        // instructions lorsque la valeur est dans liste_n
      sinon
        // instructions lorsque la valeur de la variable
        // ne se trouve dans aucune des listes précédentes
      fin si
    fin si
fin si
      \end{verbatim}
        \'Ecrivons un algorithme qui lit un jour de la semaine sous forme d'un nombre entier (1 pour
        lundi, . . ., 7 pour dimanche) et qui affiche en clair ce jour de la semaine.
      
            \par
        \begin{verbatim}
// Lit un nombre entre 1 et 7 et affiche en clair le jour de la semaine correspondant.
module jourSemaine()
    jour : entier
    lire jour
    selon que jour vaut
      1 : afficher "lundi"
      2 : afficher "mardi"
      3 : afficher "mercredi"
      4 : afficher "jeudi"
      5 : afficher "vendredi"
      6 : afficher "samedi"
      7 : afficher "dimanche"
    fin selon que
fin module
    \end{verbatim}En Java :
            \par
        \begin{verbatim}
switch (variable){
    case val1 :
      // instructions lorsque la valeur de la variable est val1
      break;
    case val2 :
    case val3 :
    case val4 :
      // instructions lorsque la valeur de la variable est val2 ou val3 ou val4
      break;
    ...
    case valN  :
      // instructions lorsque la valeur de la variable est valN
      break;
    default :
      // instructions lorsque la valeur de la variable
      // ne se trouve dans aucune des listes précédentes
}
      \end{verbatim}Notez que le cas \verb@default@ est facultatif.
            \par
        
        Notez le \verb@break@ \`a la fin de chaque (groupe de)\verb@ case@.
      
            \par
        
        La variable peut \^etre de type \verb@byte@, \verb@short@, 
        \verb@char@, \verb@int@\verb@String@ et 
        les types \'enum\'er\'es que nous verrons plus tard.
      
            \par
        
        Elle est \'equivalente \`a : 
      
            \par
        \begin{Java}
if (variable == val1){
      // instructions lorsque la valeur de la variable est val1
} else if (variable ==  val2 || variable ==  val3 || variable == val4){
      // instructions lorsque la valeur de la variable est val2 ou val3 ou val4
} else if (variable == valN){
      // instructions lorsque la valeur de la variable est valN
} else {
      // instructions lorsque la valeur de la variable
      // ne se trouve dans aucune des listes precedentes
}      \end{Java}Par exemple : 
            \par
        \begin{Java}
import java.util.Scanner;
public class Test{
  public static void main(String[] args){
      Scanner clavier = new Scanner(System.in);
      String produit = clavier.next();
      switch(produit) {
        case "Coca" :
        case "Sprite" :
        case "Fanta" :
        prixDistributeur = 60;
        break;
      case "IceTea" :
        prixDistributeur = 70;
        break;
      default :
        prixDistributeur = 0;
        break;
      }
      System.out.println(prixDistributeur);
    }
}
      \end{Java}\subsection{selon que (version avec conditions)}En pseudo-code :
            \par
        \begin{verbatim}
selon que
    condition_1 :
      // instructions lorsque la condition_1 est vraie
    condition_2 :
      // instructions lorsque la condition_2 est vraie
    ...
    condition_n :
      // instructions lorsque la condition_n est vraie
    autres :
      // instructions à exécuter quand aucune
      // des conditions précédentes n’est vérifiée
fin selon que
      \end{verbatim}
        Comme pr\'ec\'edemment, une et une seule des s\'equences d'instructions est ex\'ecut\'ee. On
        veillera \`a ce que les conditions ne se \guillemotleft  recouvrent \guillemotright  pas, c'est-\`a-dire que deux d'entre elles
        ne soient jamais vraies simultan\'ement. 
      
            \par
        C'est \'equivalent \`a : 
            \par
        \begin{verbatim}
si condition_1 alors
    // instructions lorsque la condition_1 est vraie
sinon
    si condition_2 alors
      // instructions lorsque la condition_2 est vraie
    sinon
      ...
      si condition_n alors
        // instructions lorsque la condition_n est vraie
      sinon
        // instructions à exécuter quand aucune
        // des conditions précédentes n’est vérifiée
      fin si
    fin si
fin si
      \end{verbatim}Par exemple : 
            \par
        \begin{verbatim}
// Lit un nombre et affiche si ce nombre est strictement positif , strictement négatif ou nul.
module signeNombre()
    nb : entier
    lire nb
    selon que 
      nb < 0 :
        afficher "le nombre", nb, " est négatif"
      nb > 0 :
        afficher "le nombre", nb, " est positif"
      autres : 
        afficher "le nombre", nb, " est nul"
    fin selon que
fin module
    \end{verbatim}En Java :
            \par
        Il n'existe pas de \verb@switch@ avec condition, 
		  il faut l'\'ecrire comme une succession de \verb@if@.
            \par
        \begin{Java}
if (condition_1){
      // instructions lorsque la condition_1 est vraie
} else if (condition_2){
      // instructions lorsque la condition_2 est vraie
} ... 
} else if (condition_n){
      // instructions lorsque la condition_n est vraie
} else {
      // instructions a executer quand aucune
      // des conditions precedentes n est verifiee
}
      \end{Java}Par exemple : 
            \par
        \begin{Java}
import java. util .Scanner;
public class Test {
    public static void main(String [] args) {
      Scanner clavier = new Scanner(System.in);
      int nb = clavier.nextInt();
      if (nb>0) {
        System.out. println (" positif ");
      } else if (nb==0) {
        System.out. println ("nul");
      } else {
        System.out. println (" negatif ");
      }
    }
}
    \end{Java}\section{Exercices}
				Maintenant, mettons tout \c ca en pratique.
      
            \par
        \subsection{Compr\'ehension d'algorithme}
          Pour ces exercices, nous vous demandons de comprendre des algorithmes donn\'es. 
          
			
		\subparagraph{Compr\'ehension} 
		
                \textcolor{white}{.} \par
            
							  Que vont-ils afficher ?
              
					\begin{itemize}
				
			\item \begin{verbatim}
module exerciceA()
  a,b : entier
  lire a,b
  si a > b alors
    a ← a+2*b
  fin si
  afficher a
fin module
				\end{verbatim}Si les nombres lus sont respectivement 2 et 3 ? 
            \par
         \textcolor{gray}{\underline{\hspace*{1em}}} Si les nombres lus sont respectivement 4 et 1 ? 
            \par
         \textcolor{gray}{\underline{\hspace*{1em}}} 
			\item \begin{verbatim}
module exerciceB()
    a,b,c : entier
    lire b,a
    si a > b alors
      c ← a DIV b
    sinon
      c ← b MOD a
    fin si
    afficher c
fin module
				\end{verbatim}Si les nombres lus sont respectivement 2 et 3 ? 
            \par
         \textcolor{gray}{\underline{\hspace*{1em}}} Si les nombres lus sont respectivement 4 et 1 ? 
            \par
         \textcolor{gray}{\underline{\hspace*{1em}}} 
			\item \begin{verbatim}
module exerciceC ()
    x1, x2 : entier
    ok : booléen
    lire x1, x2
    ok ← x1 > x2
    si ok alors
      ok ← ok ET x1 = 4
    sinon
      ok ← ok OU x2 = 3
    fin si
    si ok alors
      x1 ← x1 * 1000
    fin si
    afficher x1 + x2
fin module
				\end{verbatim}Si les nombres lus sont respectivement 2 et 3 ? 
            \par
         \textcolor{gray}{\underline{\hspace*{3em}}} Si les nombres lus sont respectivement 4 et 1 ? 
            \par
         \textcolor{gray}{\underline{\hspace*{3em}}} 
					\end{itemize}
				
            \par
        \subsection{Compr\'ehension de codes Java}
          Pour ces exercices, nous vous demandons de comprendre des codes Java donn\'es. 
          
			
		\subparagraph{Compr\'ehension} 
		
                \textcolor{white}{.} \par
            
							  Que vont-ils afficher si \`a chaque fois les deux nombres lus au d\'epart sont successivement 2, 3 et 4 ?
							
					\begin{itemize}
				
			\item \begin{Java}
import java.util.Scanner;
public class Exercice1 {
    public static void main(String [] args) {
        Scanner clavier = new Scanner(System.in);
        int nb1 = clavier.nextInt();
        int nb2 = clavier.nextInt();
        int nb3 = clavier.nextInt();
        if (nb1 < nb2){
          System.out.print(nb1);
        } else {
          System.out.print(nb2);
        } 
    }
}
        \end{Java} \textcolor{gray}{\underline{\hspace*{1em}}} 
			\item \begin{Java}
import java.util.Scanner;
public class Exercice2 {
    public static void main(String [] args) {
        Scanner clavier = new Scanner(System.in);
        int nb1 = clavier.nextInt();
        int nb2 = clavier.nextInt();
        int nb3 = clavier.nextInt();
        if (nb1 > nb2 && nb1 > nb3){
          System.out.print(nb1);
        } else {
            if (nb2 > nb3){
              System.out.print(nb2);
            } else {
              System.out.print(nb3);
            }
        } 
    }
}
        \end{Java} \textcolor{gray}{\underline{\hspace*{1em}}} 
			\item \begin{Java}
import java.util.Scanner;
public class Exercice3 {
    public static void main(String [] args) {
        Scanner clavier = new Scanner(System.in);
        int nb1 = clavier.nextInt();
        int nb2 = clavier.nextInt();
        int nb3 = clavier.nextInt();
        switch (nb1){
          case 1 : System.out.print("premier"); break;
          case 2 : System.out.print("deuxieme"); break;
          case 3 : System.out.print("troisieme"); break;
          default : System.out.print("pas dans le trio");
        } 
    }
}
        \end{Java} \textcolor{gray}{\underline{\hspace*{10em}}} 
			\item \begin{Java}
import java.util.Scanner;
public class Exercice3 {
    public static void main(String [] args) {
        Scanner clavier = new Scanner(System.in);
        int nb1 = clavier.nextInt();
        int nb2 = clavier.nextInt();
        int nb3 = clavier.nextInt();
        switch (nb1){
          case 1 : System.out.print("premier");
          case 2 : System.out.print("deuxieme");
          case 3 : System.out.print("troisieme");
          default : System.out.print("pas dans le trio");
        } 
    }
}
        \end{Java} \textcolor{gray}{\underline{\hspace*{20em}}} 
					\end{itemize}
				
            \par
        \subsection{\`A vous de jouer...}
          Il est temps de se lancer et d'\'ecrire vos premiers modules et programmes Java correspondant. 
          Voici quelques conseils pour vous guider dans la r\'esolution de tels probl\`emes :
          
					\begin{itemize}
				
			\item il convient d'abord de bien comprendre le probl\`eme pos\'e ; assurez-vous qu'il est parfaitement sp\'ecifi\'e ;
			\item d\'eclarez ensuite les variables (et leur type) qui interviennent dans l'algorithme ; les noms des variables risquant de ne pas \^etre suffisamment explicites ;
			\item \textbf{mettez en \'evidence les variables \guillemotleft  donn\'ees \guillemotright , les variables \guillemotleft  r\'esultats \guillemotright  et les variables de travail ;}
			\item n'h\'esitez pas \`a faire une \'ebauche de r\'esolution en fran\c cais avant d'\'elaborer l'algorithme d\'efinitif pseudo-cod\'e.
			\item \'Ecrivez la partie algorithmique \textbf{AVANT} de vous lancer dans la programmation en Java.
					\end{itemize}
				
            \par
        
        \'Ecrivez les algorithmes et codez les programmes Java correspondant qui 
          
					\begin{enumerate}
				
			\item \'etant donn\'e deux nombres quelconques, recherche et affiche le plus
              grand des deux. Attention ! On ne veut pas savoir si c'est le premier ou le deuxi\`eme qui est
              le plus grand mais bien quelle est cette plus grande valeur. Le probl\`eme est donc bien d\'efini
              m\^eme si les deux nombres sont identiques.
            
			\item \'etant donn\'e trois nombres quelconques, recherche et affiche le plus grand des trois.
			\item affiche un message indiquant si un entier est strictement n\'egatif, nul ou strictement positif.
			\item \'etant donn\'e trois nombres, recherche et affiche si le premier des 
            trois appartient \`a l'intervalle donn\'e par le plus petit et le plus grand des deux autres (bornes exclues). 
            Qu'est-ce qui change si on inclut les bornes ?
			\item \'etant donn\'e une \'equation du deuxi\`eme degr\'e, d\'etermin\'ee par le coefficient de x\texttwosuperior  , le coefficient de x et le terme ind\'ependant, 
            recherche et affiche la (ou les) racine(s) de l'\'equation (ou un message ad\'equat s'il n'existe pas de racine r\'eelle).
			\item \`a partir d'un moment exprim\'e par 2 entiers, heure et minute, affiche le moment qu'il sera une minute plus tard.
			\item v\'erifie si une ann\'ee est bissextile. Pour rappel, les ann\'ees bissextiles sont les ann\'ees multiples de 4.
             Font exception, les multiples de 100 (sauf les multiples de 400 qui sont bien bissextiles). Ainsi 2012 et 2400 sont bissextile mais pas 2010 ni 2100.
					\end{enumerate}
				
            \par
        
			
		\subparagraph{Stationnement alternatif} 
		
					\textcolor{white}{.} \par
				
          Dans une rue o\`u se pratique le stationnement alternatif, du 1 au 15 du mois, on se gare du c\^ot\'e des maisons ayant un num\'ero impair, 
          et le reste du mois, on se gare de l'autre c\^ot\'e.
          \'Ecrivez un algorithme et le code java correspondant qui, sur base de la date du jour et du num\'ero de maison devant laquelle
          vous vous \^etes arr\^et\'e, indique si vous \^etes bien stationn\'e ou non.
        
            \par
        
			
		\subparagraph{La fi\`evre monte} 
		
					\textcolor{white}{.} \par
				
          Chez l'humain la temp\'erature corporelle normale moyenne est de 37 \textdegree C (entre 36,5 \textdegree C et 37,5 \textdegree C selon les individus). 
          La fi\`evre est d\'efinie par une temp\'erature rectale au repos sup\'erieure ou \'egale \`a 38,0 \textdegree C. 
          Une fi\`evre au-del\`a de 40 \textdegree C  est consid\'er\'ee comme un risque de sant\'e majeur et imm\'ediat. 
          Lorsque la fi\`evre est mod\'er\'ee (de 37,7 \textdegree C \`a 37,9 \textdegree C), on parle de f\'ebricule.\par
				
          [Wikipedia]
        
            \par
        
          \'Ecrivez un module fi\`evre qui lit une temp\'erature au clavier et qui affiche si le patient a de la
          temp\'erature (sup\'erieure ou \'egale \`a 38,0\textdegree C) ET si cette cette fi\`evre est mod\'er\'ee (entre 38,0\textdegree C 
          et 40,0\textdegree C) ou \`a risque (strictement sup\'erieur \`a 40,0\textdegree C). 
          Rien ne doit \^etre affich\'e si le patient n'a pas de fi\`evre.
        
            \par
        \'Ecrivez le code java correspondant.
            \par
        
			
		\subparagraph{Taxes communales} 
		
					\textcolor{white}{.} \par
				
          Dans ma commune, les taxes communales des enl\`evements des immondices s'\'el\`event \`a
          
					\begin{itemize}
				
			\item 80€ pour une personne isol\'ee ;
			\item 135€ pour une famille de 2 ou 3 personnes ;
			\item 175€ pour une famille de 4 personnes ou plus.
					\end{itemize}
				
          \'Ecrivez un module qui lit le nombre de personnes composant la famille et qui affiche le prix de la taxe \`a payer.
        
            \par
        \'Ecrivez le code java correspondant.
            \par
        
			
		\subparagraph{Au cin\'ema} 
		
					\textcolor{white}{.} \par
				
          \`A Bruxelles, lors de chaque projection cin\'ematographique, une taxe de 0,5€ est
          pr\'elev\'ee sur le prix du billet de chaque spectateur.
          
					\begin{itemize}
				
			\item \'Ecrivez un module qui lit le nombre de spectateurs et qui affiche le prix de la taxe \`a payer.
			\item \'Ecrivez le code java correspondant.
			\item 
              Si le film projet\'e est un documentaire, aucune taxe n'est pr\'elev\'ee. 
              \'Ecrivez un module qui lit le nombre de spectateurs et un bool\'een (\`a vrai
              si le film est un documentaire et faux sinon) et qui affiche le prix de la taxe
              \`a payer.
            
			\item \'Ecrivez le code java correspondant.
					\end{itemize}
				
            \par
        
				\end{document}
			