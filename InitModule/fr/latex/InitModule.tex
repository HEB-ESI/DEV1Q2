\documentclass[11pt,a4paper]{article}
			\usepackage[french]{babel}
					
				\usepackage{pifont}  
				\usepackage[utf8x]{inputenc}
				\usepackage[T1]{fontenc} 
				\usepackage{lmodern}			
				\usepackage{fancyhdr}
				\usepackage{textcomp}
				\usepackage{makeidx}
				\usepackage{tabularx}
				\usepackage{multicol}
				\usepackage{multirow}
				\usepackage{longtable}
				\usepackage{color}
				\usepackage{soul}
				\usepackage{boxedminipage}
				\usepackage{shadow}
				\usepackage{framed}			
				\usepackage{array}
				\usepackage{url}
				\usepackage{ragged2e}
				\usepackage{fancybox}
				\newcommand{\cadretitre}[2]{
				  \vspace*{0.8\baselineskip}
				  \begin{center}%
				  \boxput*(0,1){%
					%\colorbox{white}{\Large\textbf{\ #1\ }}%
				  }%
				  {%
					\setlength{\fboxsep}{10pt}%
				    \Ovalbox{\begin{minipage}{.8\linewidth}\begin{center}\Large\sffamily{#2}\end{center}\end{minipage}}}%
				  \end{center}
				  \vspace*{2\baselineskip}
				  }
			
			\makeatletter
			\def\@seccntformat#1{\protect\makebox[0pt][r]{\csname the#1\endcsname\quad}}
			\makeatother

				% Permet d'afficher qqchose à une positin absolue
				\usepackage[absolute]{textpos}
				\setlength{\TPHorizModule}{1cm}
				\setlength{\TPVertModule}{\TPHorizModule}
	
				\usepackage[titles]{tocloft}
				\setlength{\cftbeforesecskip}{0.5ex}
				\setlength{\cftbeforesubsecskip}{0.2ex}
				\addto\captionsfrench{\renewcommand\contentsname{}}
				
				\usepackage[font=scriptsize]{caption}
				
				\usepackage{listings}
\lstdefinestyle{lstverb}
  {
    basicstyle=\footnotesize,
    frameround=tttt, frame=trbl, framerule=0pt, rulecolor=\color{gray},
    lineskip=-1pt,   % pour rapprocher les lignes
    flexiblecolumns, escapechar=\\,
    tabsize=4, extendedchars=true
  }
\lstnewenvironment{Java}[1][]{\lstset{style=lstverb,language=java,#1}}{}
				\ifx\pdfoutput\undefined
					\usepackage{graphicx}
				\else
					\usepackage[pdftex]{graphicx}
				\fi
				\usepackage[a4paper, hyperfigures=true, colorlinks, linkcolor=black, citecolor=blue,urlcolor=blue, pagebackref=true, bookmarks=true, bookmarksopen=true,bookmarksnumbered=true,
                pdfauthor={}, pdftitle={TD Modules}, pdfkeywords={TD Modules, },pdfpagemode=UseOutlines,pdfpagetransition=Dissolve,nesting=true,
				backref, pdffitwindow=true, bookmarksnumbered=true]{hyperref}
				\usepackage{supertabular}
				\usepackage[table]{xcolor}
				\usepackage{url}
				\usepackage{caption} 
				\setlength{\parskip}{1.3ex plus 0.2ex minus 0.2ex}
				\setlength{\parindent}{0pt}
				
				\makeatletter
				\def\url@leostyle{ \@ifundefined{selectfont}{\def\UrlFont{\sf}}{\def\UrlFont{\footnotesize\ttfamily}}}
				\makeatother
				\urlstyle{leo}
				
				\definecolor{examplecolor}{rgb}{0.156,0.333,0.443}
				\definecolor{definitioncolor}{rgb}{0.709,0.784,0.454}
				\definecolor{exercisecolor}{rgb}{0.49,0.639,0}
				\definecolor{hintcolor}{rgb}{0.941,0.674,0.196}
				\definecolor{tableHeadercolor}{rgb}{0.709,0.784,0.454}
				\definecolor{tablerowAltcolor}{rgb}{.866,.905,.737}
				\definecolor{tablerowAlt2color}{rgb}{.968,.976,.933}
				\definecolor{verylightgray}{rgb}{0.98,0.98,0.98}
				
				\newenvironment{fshaded}{
				\def\FrameCommand{\fcolorbox{framecolor}{shadecolor}}
				\MakeFramed {\FrameRestore}}
				{\endMakeFramed}
				
				\newenvironment{fexample}[1][]{\definecolor{shadecolor}{rgb}{.913,.913,.913}
				\definecolor{framecolor}{rgb}{.156,.333,.443}
				\begin{fshaded}}{\end{fshaded}} 
				
				\newenvironment{fdefinition}{\definecolor{shadecolor}{rgb}{.913,.913,.913}
				\definecolor{framecolor}{rgb}{.709,.784,.454}
				\begin{fshaded}}{\end{fshaded}}
				
				\newenvironment{fexercise}{\definecolor{shadecolor}{rgb}{.913,.913,.913}
				\definecolor{framecolor}{rgb}{.49,.639,0}
				\begin{fshaded}}{\end{fshaded}}
				
				\newenvironment{fhint}{\definecolor{shadecolor}{rgb}{.913,.913,.913}
				\definecolor{framecolor}{rgb}{.941,.674,.196}
				\begin{fshaded}}{\end{fshaded}}	
				
				\newcommand{\PreserveBackslash}[1]{
				\let\temp=\\#1\let\\=\temp
				}
				\let\PBS=\PreserveBackslash
				\newcolumntype{A}{>{\PBS\raggedright\small\hspace{0pt}}X}
				\newcolumntype{L}[1]{>{\PBS\raggedright\small\hspace{0pt}}p{#1}}
				\newcolumntype{R}[1]{>{\PBS\raggedleft\small\hspace{0pt}}p{#1}}
				\newcolumntype{C}[1]{>{\PBS\centering\small\hspace{0pt}}p{#1}}
				
				\makeindex
				
				\title{TD Modules}	
			\date{}
			\author{\scriptsize{}}
			\definecolor{light-gray}{gray}{0.8}
			\renewcommand{\headrulewidth}{0pt}
			\fancyhead[L]{
				\footnotesize\textsc{Haute \'Ecole de Bruxelles}\\
	    			\footnotesize\textsc{\'Ecole Sup\'erieure d'Informatique}
			}
			\fancyhead[R]{
				\footnotesize{Bachelor en Informatique}\\
				\footnotesize{Laboratoires Java} - 
			\footnotesize{1\`ere ann\'ee}}
				\fancyfoot[L]{ }
				\fancyfoot[C]{}
				\fancyfoot[R]{\scriptsize{\textcolor{gray}{version 2014-2015 (\today)}}}
				\pagestyle{plain}
				\reversemarginpar
				\usepackage{rotating}						
				\begin{document}
					\begin{textblock}{9}(2,3.2)
						\includegraphics[width=2cm]{../../../_templates/java/icons/logo-esi}
					\end{textblock}
				
				
				
				
				%\maketitle
				\cadretitre{TD1}{TD Modules}
				\thispagestyle{fancy}
        \marginpar{\begin{sideways}
            \begin{minipage}[t]{1cm}
            \begin{tiny}
            \includegraphics[width=1\linewidth,height=1\textheight,keepaspectratio=true]{../../../_templates/java/icons/cc-gris.jpg}
			\end{tiny}
			\end{minipage}
            \begin{minipage}[b]{19cm}
            \begin{tiny}
            \textcolor{gray}{Distribué sous licence Creative Commons Paternité - Partage à l'Identique 2.0 Belgique 
            (\texttt{http://creativecommons.org/licenses/by-sa/2.0/be/})
			\vspace{-1em}
			\\Les autorisations au-delà du champ de cette licence peuvent être obtenues à 
			\texttt{http://www.heb.be/esi}
			- \texttt{mcodutti@heb.be}
			}\end{tiny}
			\end{minipage}
        \end{sideways}}
            \begin{abstract}
			Ces exercices ont pour but de v\'erifier que vous avez fix\'e comment d\'ecouper un algorithme en modules
      (morceaux d'algorithmes) et en m\'ethodes (morceaux de code).
		
            \par
        \end{abstract}
				\vspace{-2em}\tableofcontents
				\pagestyle{plain}
            \clearpage
            \fancyhead[L,C,R]{}
            \fancyfoot[L,C]{}
            \fancyfoot[R]{ \scriptsize{\textcolor{gray}{
				InitModule - page \thepage}}}
				\thispagestyle{fancy}
				\pagestyle{fancy}
	   
            \section{D\'ecoupe du code}\subsection{\`A vous de jouer...}
          Voici quelques conseils pour vous guider dans la r\'esolution de tels probl\`emes :
          
					\begin{itemize}
				
			\item il convient d'abord de bien comprendre le probl\`eme pos\'e ; assurez-vous qu'il est parfaitement sp\'ecifi\'e ;
			\item r\'esolvez le probl\`eme via quelques exemples pr\'ecis ;
			\item mettez en \'evidence les variables \textbf{\guillemotleft  donn\'ees \guillemotright }, les variables \textbf{\guillemotleft  r\'esultats \guillemotright } et les variables de travail ;
			\item n'h\'esitez pas \`a faire une \'ebauche de r\'esolution en fran\c cais avant d'\'elaborer l'algorithme d\'efinitif pseudo-cod\'e ;
			\item d\'eclarez ensuite les variables (et leur type) qui interviennent dans chaque module ; les noms des variables risquant de ne pas \^etre suffisamment explicites.
			\item \'Ecrivez la partie algorithmique \textbf{AVANT} de vous lancer dans la programmation en Java.
					\end{itemize}
				
            \par
        
			
		\subparagraph{What time is it?} 
		
					\textcolor{white}{.} \par
				
          Le syst\`eme horaire anglais est bas\'e sur un cycle de 12 heures (\guillemotleft  12-hour clock \guillemotright ) au lieu du syst\`eme continental utilisant un cycle de 24 heures. 
          Pour distinguer les moments de l'avant-midi et de l'apr\`es-midi, on utilise les abr\'eviations AM (ante-meridiem) pour tout moment entre minuit et midi 
          et PM (post-meridiem) pour tout moment entre midi et minuit. 
          Les heures sont repr\'esent\'ees par les nombres de 1 \`a 12 (et jamais par 0, qui n'est pas utilis\'e pour les heures). 
          Il y a 2 cas sp\'eciaux : comme midi ne peut \^etre avant ou apr\`es lui-m\^eme, son \'ecriture correcte dans le syst\`eme anglais est \guillemotleft  12 noon \guillemotright , de m\^eme que minuit s'\'ecrit \guillemotleft  12 midnight \guillemotright . 
          Il est donc incorrect d'\'ecrire 12 AM ou 12 PM. 
        
            \par
        
          Mettez en \'evidence les variables \textbf{\guillemotleft  donn\'ees \guillemotright }, 
          les variables \textbf{\guillemotleft  r\'esultats \guillemotright } et les variables de travail ;
        
            \par
        
          \'Ecrivez un algorithme qui re\c coit en param\`etre des heures et des minutes et affiche l'heure correspondante au format anglais. (Dans cet exercice, on ne tient donc pas compte des secondes.)
        
            \par
        
          Exemples : 	
          
					\begin{itemize}
				
			\item 0h18 devient 12 :18 AM
			\item 8h15 devient 8 : 15 AM
			\item 12h00 devient 12 noon
			\item 12h30 devient 12 : 30 PM
			\item 13h25 devient 1 : 25 PM
			\item 23h42 devient 11 :42 PM
			\item 0h00 devient 12 midnight
					\end{itemize}
				
            \par
        
          Si les nombres entr\'es ne sont pas dans les bornes accept\'ees pour les heures et les minutes, vous lancerez une erreur qui arr\^etera le programme.
        
            \par
        
          \'Ecrivez le code java correspondant avec la documentation javadoc pour chaque m\'ethode. 
          N'oubliez pas d'\'ecrire la m\'ethode \verb@main@ qui fera appel \`a votre m\'ethode.
        
            \par
        
			
		\subparagraph{La plus grande somme} 
		
					\textcolor{white}{.} \par
				
          \'Ecrire un algorithme qui re\c coit deux nombres entiers positifs de 5 chiffres et retourne celui dont la somme des chiffres est la plus grande.
          Mettez en œuvre tout ce qu'il faut pour ne pas devoir r\'ep\'eter plusieurs fois le m\^eme code !
        
            \par
        
          Si les nombres entr\'es ne sont pas positifs, vous lancerez une erreur qui arr\^etera le programme.
        
            \par
        
          Par exemple, si les nombres entr\'es sont 11789 et 50211, l'algorithme retourne 11789.
        
            \par
        
          Mettez en \'evidence les variables \textbf{\guillemotleft  donn\'ees \guillemotright }, 
          les variables \textbf{\guillemotleft  r\'esultats \guillemotright } et les variables de travail ;
        
            \par
        
          \'Ecrivez le code java correspondant avec la documentation javadoc pour chaque m\'ethode. 
          N'oubliez pas d'\'ecrire la m\'ethode \verb@main@ qui fera appel \`a votre m\'ethode.
        
            \par
        
			
		\subparagraph{La boule de cristal} 
		
					\textcolor{white}{.} \par
				
          Cet algorithme est destin\'e \`a pr\'edire l'avenir, et il doit \^etre infaillible ! 
          Il recevra un moment de la journ\'ee en param\`etre (sous forme  de trois entiers : heure, minute et seconde) et affichera le moment qu'il sera une minute plus tard ! 
        
            \par
        
          Par exemple, si le moment entr\'e est 21h 32' 45'', l'algorithme affichera 21h 33' 45''.
        
            \par
        
          Si les nombres entr\'es ne sont pas dans les bornes accept\'ees pour les heures, les minutes et les secondes, vous lancerez une erreur qui arr\^etera le programme. 
        
            \par
        
          Si vous avez d\'ej\`a fait ce test pour les heures et les minutes, avez-vous pens\'e \`a modulariser ? Si pas, c'est le moment de le faire :)
        
            \par
        
          Mettez en \'evidence les variables \textbf{\guillemotleft  donn\'ees \guillemotright }, 
          les variables \textbf{\guillemotleft  r\'esultats \guillemotright } et les variables de travail ;
        
            \par
        
          \'Ecrivez le code java correspondant avec la documentation javadoc pour chaque m\'ethode. 
          N'oubliez pas d'\'ecrire la m\'ethode \verb@main@ qui fera appel \`a votre m\'ethode.
        
            \par
        Pour plus d'exercices, 
        r\'evisez ici (\url{www.heb.be/esi/InitModule/fr/../../TDModule/fr/html/Exercices\_learningObject3.html})
            \par
        
				\end{document}
			